%TODO: sposterei questa sezione a dopo o prima della sezione Soluzioni (dipende da cos'è Soluzione)
\section{Tool di configurazione}
\subsection{Obiettivi}
Questi tool di configurazione hanno lo scopo di fornire un'interfaccia client 
per gestire e interagire con un cluster distribuito. Gli obiettivi principali includono:
\begin{itemize}
  \item Gestione centralizzata: consentono agli utenti di controllare la configurazione 
      e lo stato di un cluster, e di eseguire operazione su di esso da un'unica interfaccia 
        a riga di comando (CLI), facilitando operazioni 
  amministrative come il cambio di configurazioni, e utente come la richiesta di files.
  \item Semplicità di comunicazione: Progettato per sostituire gli scipt per nulla semplici 
      e troppo meccanici e per essere user-friendly con una CLI che permette agli amministratori 
        di sistema di eseguire comandi complessi tramite semplici input.
\end{itemize}


\subsection{Funzionamento}
\subsubsection{Admin}
All'avvio della console, vengono inizializzati i vari componenti, configurato l'ambiente e 
avviati tutti i servizi 
necessari per il funzionamento. Durante questa fase, vengono effettuate operazioni importanti 
come la configurazione del cluster, l'impostazione dell'interfaccia CLI, che consente agli 
amministratori di lanciare comandi specifici, e la connessione con altri servizi o nodi distribuiti.

Una delle componenti principali è vmManager, incaricata della gestione delle macchine virtuali (VM).
Questi si occupa della creazione, modifica e cancellazione delle VM, nonché della loro 
configurazione, che viene gestita tramite file XML, come vol.xml, dom.xml e pool.xml. 
Questi file contengono le informazioni relative alle risorse di virtualizzazione assegnate alle 
VM, come CPU, memoria e spazio disco. Infine, implementa le logiche necessarie per interfacciarsi 
con l'hypervisor, consentendo una gestione dinamica delle risorse virtualizzate.

La gestione dei nodi viene invece affidata ad un'altra componente chiamata clusterManager. 
In particolare, ne gestisce le connessioni e le configurazioni, inviandone di nuove. 

Il sistema permette agli amministratori di interagire tramite una CLI, la quale offre la 
possibilità di aggiornare le configurazioni del cluster, monitorare lo stato dei nodi e gestire 
la topologia del sistema aggiungendo o 
rimuovendo nodi. La CLI utilizza modelli di dati per garantire che tutte le informazioni inserite 
siano strutturate e validate correttamente.

Le risposte della console variano a seconda dell'operazione richiesta: nel caso di modifica della 
configurazione, la risposta sarà una conferma, mentre, se si vuole conoscere lo stato del cluster, 
l'applicativo risponderà con un elenco dei nodi attivi.

\subsubsection{Client}
All'avvio, il client viene configurato per connettersi al cluster di nodi distribuiti. In seguito, 
l'utente interagisce con il client tramite un'interfaccia a riga di comando (CLI), che mette a 
disposizione diversi operazioni eseguibili sui diversi nodi del cluster, come ottenere, modificare 
o eliminare un particolare file. I comandi inseriti dall'utente vengono interpretati e parsati, 
per determinare quale azione deve essere eseguita, il client prepara una richiesta da inviare al 
cluster e questa viene serializzata e inviata al nodo di destinazione, che potrebbe essere il 
leader del cluster o un nodo follower. Nel secondo caso, il nodo follower invierà in risposta 
l'indirizzo del leader, in alternativa si attende una risposta del cluster all'operazione eseguita.

Una volta ricevuta la risposta dal cluster, questa viene deserializzata e il risultato 
dell'operazione viene poi presentato all'utente tramite l'interfaccia CLI. Ad esempio, se l'utente 
ha richiesto di leggere un file, il client mostrerà il file richiesto, mentre per operazioni di 
scrittura, modifica ed eliminazione, la risposta sarà una semplice conferma.

\subsection{Limiti}
Nonostante i suoi punti di forza, i tool presentano alcuni limiti:
\begin{itemize}
  \item Mancanza di un'interfaccia grafica: l'uso esclusivo di una CLI potrebbe risultare poco 
      intuitivo per utenti non tecnici o per chi preferisce un'interfaccia grafica.
  \item Manutenzione complessa del cluster: in caso di aggiornamenti di configurazione complessi, 
      il sistema potrebbe richiedere un intervento manuale sul cluster.
  \item Gestione limitata degli errori: in situazioni di errori prolungati o corruzione di dati, 
      il tool potrebbe non essere sufficiente a garantire un recupero automatico e sicuro.
\end{itemize}
In sintesi, questi tool di configurazione sono stati progettati per facilitare la comunicazione 
con il cluster, ma potrebbero richiedere ulteriori funzionalità per migliorare la resilienza e 
l'usabilità in ambienti complessi.
