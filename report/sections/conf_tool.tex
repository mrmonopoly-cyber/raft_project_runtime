\section{Configuration Tools}
\subsection{Objectives}
The purpose of these configuration tools is to provide a client interface for 
managing and interacting with a distributed cluster. The main objectives include:

\begin{itemize}
  \item \textbf{Centralized Management}: These tools allow users to control 
    the configuration and state of a cluster, as well as perform operations 
    on it through a single Command Line Interface (CLI). This simplifies 
    administrative tasks, such as configuration changes, and user operations,
    such as file requests.
  \item \textbf{Ease of Communication}: Designed to replace overly complex and 
    mechanical scripts, these tools aim to be user-friendly. The CLI enables 
    system administrators to execute complex commands through simple inputs, 
    improving usability and reducing the chance of errors.
\end{itemize}

\subsection{Functionality}
\subsubsection{Admin}
When the console starts, various components are initialized, the environment 
is configured, and all necessary services for operation are launched. During 
this phase, key tasks are performed, such as setting up the cluster 
configuration, initializing the CLI interface to allow administrators to issue 
specific commands, and establishing connections with other distributed 
services or nodes.

One of the main components is the \texttt{vmManager}, responsible for managing 
virtual machines (VMs). It handles the creation, modification, and deletion of 
VMs, as well as their configuration through XML files such as \texttt{vol.xml}, 
\texttt{dom.xml}, and \texttt{pool.xml}. These files define the virtualization 
resources assigned to the VMs, including CPU, memory, and disk space. 
Additionally, it implements the necessary logic to interface with the 
hypervisor, enabling dynamic management of virtualized resources.

Node management, on the other hand, is handled by another component called 
\texttt{clusterManager}. This component manages node connections and 
configurations, distributing updates as needed.

The system provides administrators with a CLI, offering the ability to update 
cluster configurations, monitor node statuses, and manage the system topology 
by adding or removing nodes. The CLI uses data models to ensure that all 
input is structured and correctly validated.

The console's responses vary depending on the requested operation. For 
configuration changes, the response is a confirmation, while for status 
inquiries, the application responds with a list of active nodes.
\subsubsection{Client}
At startup, the client is configured to connect to the distributed cluster
of nodes. The user interacts with the client through a Command Line Interface
(CLI), which provides various operations that can be performed on the 
cluster's nodes, such as retrieving, modifying, or deleting specific files. 

Commands entered by the user are interpreted and parsed to determine the 
required action. The client then prepares a request to send to the cluster. 
This request is serialized and transmitted to the target node, which could 
either be the cluster leader or a follower node. In the latter case, the 
follower node responds with the leader's address. Otherwise, the cluster 
processes the operation and provides a response.

Once a response is received from the cluster, it is deserialized, and the 
result of the operation is presented to the user via the CLI. For instance, 
if the user requests to read a file, the client displays the requested file. 
For write, modify, or delete operations, the response is typically a 
confirmation message.

\subsection{Limitations}
Despite its strengths, the tools have some limitations:

\begin{itemize}
  \item \textbf{Lack of a Graphical Interface}: The exclusive use of a CLI 
    may be unintuitive for non-technical users or those who prefer 
    graphical interfaces.
  \item \textbf{Complex Cluster Maintenance}: For complex configuration 
    updates, the system might require manual intervention on the cluster.
  \item \textbf{Limited Error Handling}: In cases of prolonged errors or 
    data corruption, the tool may not guarantee automatic and secure recovery.
\end{itemize}
In summary, these configuration tools are designed to facilitate communication 
with the cluster but may require additional features to improve resilience 
and usability in complex environments.
