\section{Project organization}
Now that the objectives of our project have been broadly defined, we can 
describe how the project itself was structured.
Using Git, three main branches were created:
\begin{itemize}
  \item \texttt{main}: Serves as the interface for all project functionalities, 
    from installation to usage. It provides a set of programs and scripts for these purposes.

  \item \texttt{RaftDev}: Contains only the stable version of the code used by the
    nodes, defining their synchronization and the provided functionalities.

  \item \texttt{raft\_executables}: Contains a compiled version of the code. This is 
    downloaded by the nodes during the execution of the \texttt{raft\_daemon} service 
    (a service present in the operating system of the VMs).
    The rationale behind this unique choice lies in achieving a more seamless 
    and straightforward way to update the code executed by the VMs, enabling 
    hot updates.
    This approach was particularly helpful during development phases, allowing 
    testing of different code versions. In this branch, a script is included 
    that compiles the code from a specific branch (provided as input) and saves
    the binary in a directory that the VM will then execute.
    This allowed work on multiple features in parallel without needing to 
    redefine the VMs for code testing.

  \item others: Any other branch is dedicated to the development of new features. 
    Once the development is complete, a \textbf{merge} procedure with 
    \texttt{RaftDev} will be carried out.
\end{itemize}
