\section{Organizzazione di progetto}
Ora che sono stati definiti, a grandi linee, gli obbiettivi del nostro progetto possiamo
decrivere come e' stato strutturato il progetto. 
Utilizzando git sono stati creati tre branch principali:
\begin{itemize}
    \item main: funge da interfaccia per tutte le funzionalita' del progetto:
                dall'installazione all'utilizzo. Fornisce un insieme di programmi e script
                per lo scopo.
    \item RaftDev:  contiene solamente la versione stabile del codice che viene utilizzato dai nodi
                    definendo la sincronizzazione degli stessi e le funzionalita' fornite.
    \item raft\_executables: contiene una versione compilata del codice. Viene scaricata dai nodi
                            ogni qualvolta il servizio raft\_daemon. La ragione di questa 
                            scelta singolare risiede in una maggiore e piu' semplice capacita' di
                            aggiornamento del codice eseguito dalle VM permettendo un aggiornamento
                            a caldo delle stesse. Ha aiutato molto nelle fasi di sviluppo per testare
                            diverse versioni del codice poiche' in questo branch e' presente uno script
                            che, data in input il nome di uno specifico branch, compila il codice
                            e salva il binario im una cartella che la VM poi eseguira'a.
                            Cosifacendo e' stato possibile lavorare su diverse featues in parallelo
                            senza dover ridefinire le VM per poter testare il codice.
    \item altri: qualunque altro branch e' un branch di sviluppo dedito alla implementazione di una
                 nuova funzionalita'.
\end{itemize}
