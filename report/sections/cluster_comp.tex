
\section{Composizione di un cluster}
Prima di poter analizzare i nostri obiettivi, come li abbiamo raggiunti e perché siano state fatte determinate scelte, è necessario descrivere gli elementi che compongono un \textit{cluster} e le loro caratteristiche:

\begin{itemize}
  \item \textbf{Nodi}: Sono i computer che si occupano della computazione necessaria per mantenere il servizio attivo. Possono essere fisici (\textit{server} dedicati) o virtuali (macchine virtuali, \textit{container}).
  \item \textbf{Gestore dei nodi}: Questo componente si occupa dell'aggiunta, rimozione e sostituzione dei nodi nel \textit{cluster}. Può essere un \textit{hypervisor}, nel caso i nodi siano virtuali, o un operatore, nel caso i nodi siano server reali. È probabile che ci siano più gestori, soprattutto se i nodi sono geograficamente distanti.
  \item \textbf{Gestore di rete}: Questo componente gestisce le connessioni interne ed esterne al \textit{cluster}. In particolare, si occupa dell'assegnazione degli indirizzi di rete ai nodi (è possibile che ogni nodo abbia più interfacce di rete e quindi più indirizzi IP).
  \item \textbf{Sistema operativo}: Sebbene appartenga al nodo, il sistema operativo deve essere progettato su misura per il \textit{cluster}, poiché anche questo componente è responsabile per la corretta operatività dell'intera infrastruttura. In particolare, dovrà:
    \begin{itemize}
        \item Ottenere gli indirizzi di rete (è possibile siano più di uno per ogni nodo).
        \item Eseguire una diagnostica interna sullo stato della macchina.
        \item Eseguire eventuali programmi o daemon responsabili delle funzionalità del \textit{cluster} e degli applicativi.
    \end{itemize}
\end{itemize}
