\section{A cluster composition}
Before analyzing our goals, how we achieved them, and the rationale behind certain 
choices, it is necessary to describe the components of a cluster and their characteristics:
\begin{itemize}
	\item \textbf{Nodes}: These are the computers responsible for performing the computations
    needed to keep the service running. They can be physical (dedicated servers) or 
    virtual (virtual machines or containers).
	
	\item \textbf{Node Manager}: This component handles the addition, removal, and 
    replacement of nodes within the cluster. It may be an hypervisor if the nodes are 
    virtual or an operator if the nodes are physical servers. In cases where nodes 
    are geographically dispersed, multiple node managers may be required.	
	
	\item \textbf{Network Manager}: This component manages both internal and external 
    connections to the cluster. Specifically, it handles assigning network addresses 
    to nodes. It is possible for each node to have multiple network interfaces, and 
    therefore multiple IP addresses.
	
	\item \textbf{Operating System}: Although part of the node, the operating system 
    must be tailored to the cluster because it plays a crucial role in the correct 
    functioning of the entire infrastructure. In particular, it must:
	\begin{itemize}
		\item Obtain network addresses (possibly more than one per node).
		\item Perform internal diagnostics on the machine's status.
		\item Run any programs or daemons responsible for the cluster's functionality 
      and its applications.
	\end{itemize}
\end{itemize}
