\usepackage{listings}

\section{PIB}
\subsection{Elementi}
\subsubsection{Nodi}
Come gia' accennato i nodi del cluster sono dell \textbf{Virtual Machine(VM)} create trammite 
\textbf{libvirt} usando il seguente file XML che ne determina le caratteristiche:
\\
\\
\\
\begin{lstlisting}[language=XML]
<domain type='kvm' id='1'>\\
  <name>RAFT\_NODE\_NAME</name>\\
  <metadata>\\
    <libosinfo:libosinfo xmlns:libosinfo=\\"http://libosinfo.org/xmlns/libvirt/domain/1.0">\\
      <libosinfo:os id="http://archlinux.org/archlinux/rolling"/>\\
    <\/libosinfo:libosinfo>\\
  </metadata>\\
  <memory unit='KiB'>1048576</memory>\\
  <currentMemory unit='KiB'>1048576</currentMemory>\\
  <vcpu placement='static'>1</vcpu>\\
  <resource>\\
    <partition>/machine</partition>\\
  </resource>\\
  <os>\\
    <type arch='x86\_64' machine='pc-q35-8.2'>hvm</type>\\
  </os>\\
  <features>\\
    <acpi/>\\
    <apic/>\\
    <vmport state='off'/>\\
  </features>\\
  <cpu mode='host-passthrough' check='none' migratable='on'/>\\
  <clock offset='utc'>\\
    <timer name='rtc' tickpolicy='catchup'/>\\
    <timer name='pit' tickpolicy='delay'/>\\
    <timer name='hpet' present='no'/>\\
  </clock>\\
  <on\_poweroff>destroy</on\_poweroff>\\
  <on\_reboot>restart</on\_reboot>\\
  <on\_crash>destroy</on\_crash>\\
  <!-- Devices section -->\\
  <devices>\\
    <emulator>/usr/bin/qemu-system-x86\_64</emulator>\\
    <disk type='file' device='disk'>\\
      <driver name='qemu' type='qcow2' discard='unmap'/>\\
      <source file='PATH\_DISK'/>\\
      <target dev='vda' bus='virtio'/>\\
      <boot order='1'/>\\
    </disk>\\
    <disk type='file' device='cdrom'>\\
      <driver name='qemu' type='raw'/>\\
      <source file='/var/lib/libvirt/images/raft\_live\_install.iso'/>\\
      <target dev='sda' bus='sata'/>\\
      <readonly/>\\
      <boot order='2'/>\\
    </disk>\\
    <!-- Other devices omitted for brevity -->\\
  </devices>\\
</domain>\\
\end{lstlisting}

Come precisato nel file di qui sopra ogni nodo e' composto da 1048576KB (1024 MB) di memoria,
ha ua'architettura x86\_64, ha un disco virtuale (file \textbf{.qcow2}) la cui \textbf{PATH} 
non e' specificata in quanto parametrica e determinata al momento dell'allocazione del nodo.
Inoltre a ogni nodo viene associata l'immagine di sistema \textbf{raft\_live\_install.iso} che 
contiene il sistema operativo. il parametro \textbf{RAFT\_NODE\_NAME} e' anch'esso determinato
a tempo di creazione del nodo e per questo risulta, in questo file XML, parametrico.

Prima dell'allocazione di un nodo verranno creati: un XML temporaneo con le caratteristiche 
specifiche del nuovo nodo e un disco virtuale da associare al nodo. 
Una volta generati questi file \textbf{libvirt} creera' una VM seguendo le specifiche del 
nuovo file XML lasciando inalterato l'originale che continuera' ad essere utilizzato come template.

Tale approccio ci permette di generalizzare e personalizzare le specifiche dei nodi garantendone
un semplice e preciso controllo.


% Sezione vuota per scaletta

\subsubsection{Gestore dei nodi e della rete}
% Sezione vuota per scaletta
Abbiamo progettato l'architettura di rete del \textit{cluste}r con due \textit{subnet} distinte, ciascuna con ruoli specifici per garantire la separazione tra il traffico pubblico rivolto ai \textit{client} e la comunicazione interna del \textit{cluster}:
\begin{itemize}
  \item \textit{Subnet} Pubblica (192.168.2.0/24): Questa \textit{subnet} è basata su \textit{NAT}, il che significa che i nodi nella rete pubblica possono accedere a \textit{internet} esterno e comunicare con sistemi esterni, 
    ma non possono comunicare direttamente tra loro all'interno della \textit{subnet}. Il \textit{pool} di rete pubblica è riservato ai nodi che devono interagire con servizi o client esterni. In questo intervallo, 
    gli indirizzi 1 e 255 sono riservati rispettivamente all'\textit{hypervisor} e ai messaggi di \textit{broadcast}, mentre i restanti indirizzi IP sono assegnati dinamicamente alle macchine virtuali che necessitano di 
    accesso esterno.
  \item \textit{Subnet} Privata (10.0.0.0/24): Questa \textit{subnet} è utilizzata esclusivamente per la comunicazione \textit{intra-cluster}. L'abbiamo creata utilizzando un \textit{bridge} virtuale, che consente a tutti i 
    nodi della \textit{subnet} di comunicare direttamente tra loro. Tuttavia, questa rete non ha accesso a \textit{internet} esterno, assicurando che il suo unico scopo sia facilitare la comunicazione tra i nodi per compiti 
    correlati al \textit{cluster}. Come nella \textit{subnet} pubblica, gli indirizzi 1 e 255 sono riservati, e il resto dell'intervallo è disponibile per i nodi del \textit{cluster}.
\end{itemize}

Separando le \textit{subnet} pubbliche e private, garantiamo che le richieste dei \textit{client} e i messaggi \textit{intra-cluster} siano instradati correttamente senza interferenze. La \textit{subnet} pubblica gestisce 
le interazioni con i \textit{client}, mentre la \textit{subnet} privata è dedicata alle operazioni interne, come il coordinamento e la replica dello stato tra i nodi, come richiesto dal protocollo \textit{Raft}. Questa separazione 
non solo migliora la sicurezza e le prestazioni, ma facilita anche la manutenibilità e la scalabilità del \textit{cluster}.

\subsubsection{Sistema operativo}
% Sezione vuota per scaletta
Nel nostro \textit{cluster}, ogni macchina virtuale esegue \textit{Arch Linux} come sistema operativo. All'avvio, due processi chiave vengono avviati automaticamente su ogni nodo.
\begin{itemize}
  \item \textit{\textbf{raft\_daemon.service}}: Questo processo è responsabile dell'esecuzione del codice relativo al protocollo di consenso \textit{Raft}, che garantisce che il nostro sistema distribuito mantenga la tolleranza 
    ai guasti e la consistenza tra tutti i nodi.
  \item \textit{\textbf{discovery.service}}: Questo processo facilita la scoperta degli indirizzi IP degli altri nodi all'interno della \textit{subnet} 10.0.0.x/24. Lo fa eseguendo scansioni di rete periodiche utilizzando 
    \textit{nmap}. Ogni 30 secondi, il processo di \textit{discovery} recupera e memorizza gli indirizzi IP delle interfacce di rete pubbliche e private degli altri nodi. Questa ricerca continua garantisce che ogni nodo 
    rimanga "consapevole" degli altri nodi all'interno della \textit{subnet}, mantenendo così una comunicazione continua tra loro.
\end{itemize}



\subsection{Raft per il consenso distribuito}
\textit{Raft} è un algoritmo di consenso distribuito, ovvero un meccanismo che permette a un gruppo di computer (nodi) di raggiungere un accordo su un dato stato, anche in presenza di guasti o latenze di rete. In pratica, 
\textit{Raft} assicura che tutti i nodi abbiano una copia identica e aggiornata dei dati, garantendo così la coerenza e l'affidabilità di un sistema distribuito.

\subsubsection{Motivazioni}
% Sezione vuota per scaletta
Viene spesso preferito per il consenso distribuito per diverse ragioni:
\begin{itemize}
  \item \textbf{Semplicità}: \textit{Raft} è stato progettato per essere più comprensibile rispetto ad altri algoritmi di consenso. Un esempio è \textit{Paxos}, che ha acquisito una reputazione per la sua difficoltà sia 
  nell'implementazione che nella comprensione. È proprio questa semplicità che lo rende accessibile agli sviluppatori.
  
  \item \textbf{Tolleranza ai Guasti}: In ambienti distribuiti, è molto probabile che uno o più nodi si guastino. \textit{Raft}, tramite i suoi meccanismi, può garantire che il sistema continuerà a funzionare correttamente 
  anche se alcuni nodi non sono attivi o non rispondono.

  \item \textbf{Replica Uniforme dei Dati}: \textit{Raft} assicura che i dati siano replicati uniformemente tra i nodi, in modo che essi mantengano copie esatte dello stato condiviso, il che assicura alta disponibilità e resilienza.

  \item \textbf{Leadership Chiaramente Definita}: A differenza di \textit{Paxos}, \textit{Raft} semplifica la gestione del consenso eleggendo un \textit{leader} tra i nodi. Il \textit{leader} organizza la replica dei dati 
  e coordina le decisioni critiche, riducendo la complessità dell'algoritmo.

  \item \textbf{Efficienza}: \textit{Raft} può raggiungere il consenso in tempi relativamente brevi; pertanto, è adatto per applicazioni che richiedono velocità e bassa latenza, riducendo i ritardi nella replica dei dati.

\end{itemize}


\subsubsection{Applicazione}
% Sezione vuota per scaletta
Raft si applica a una vasta gamma di applicazioni distribuite, specialmente in scenari che richiedono alta disponibilità, tolleranza ai guasti e coerenza dei dati. Esempi di applicazioni includono:

\begin{itemize}
  \item \textbf{Database Distribuiti}: \textit{Raft} è utilizzato, come in \textit{Etcd} e \textit{TiKV}, per replicare i dati in modo consistente su più nodi, garantendo che, anche in caso di guasti \textit{hardware} o di rete, 
  nessun dato venga perso.

 \item \textbf{Sistemi di File Distribuiti}: Per sistemi che distribuiscono \textit{file} su vari nodi, come \textit{HDFS}, \textit{Raft} garantisce che le modifiche ai \textit{file} siano applicate in modo coerente mantenendo 
 il sistema disponibile durante i guasti dei nodi.

 \item \textbf{Sistemi di Coordinamento e \textit{Registry}}: Il protocollo \textit{Raft} è utilizzato in servizi come \textit{Consul} ed \textit{Etcd} per la configurazione distribuita e la scoperta dei servizi. Permette a un 
 \textit{cluster} di nodi di raggiungere il consenso sui cambiamenti di stato in modo affidabile.

 \item \textbf{\textit{Blockchain} Private}: Alcune implementazioni di \textit{blockchain} private, come \textit{Hyperledger}, utilizzano \textit{Raft} poiché raggiungere rapidamente il consenso con tolleranza ai guasti per 
 la validazione delle transazioni è fondamentale.

 \item \textbf{\textit{Cache} Distribuite}: Nei sistemi di \textit{caching} distribuito come \textit{Redis} in modalità \textit{cluster}, \textit{Raft} può essere utilizzato per garantire che gli aggiornamenti 
 effettuati alla \textit{cache} siano propagati in modo coerente tra i nodi del \textit{cluster}.
\end{itemize}

\subsubsection{Limiti}
% Sezione vuota per scaletta
Sebbene Raft presenti molti vantaggi, ci sono alcune limitazioni da considerare durante il processo di progettazione e implementazione:
\begin{itemize}
  \item \textbf{Scalabilità Limitata}: Sebbene \textit{Raft} sia molto più semplice rispetto agli altri algoritmi di consenso, soffre di problemi di scalabilità in grandi \textit{cluster}. Un \textit{leader} centralizzato limita 
  la capacità di scrittura alla velocità di elaborazione del \textit{leader} stesso. Questo può diventare un collo di bottiglia in ambienti con centinaia o migliaia di nodi.

  \item \textbf{Dipendenza dal \textit{Leader}}: Il meccanismo di elezione del \textit{leader} può risultare problematico se il \textit{leader} cambia frequentemente, come in ambienti instabili. Ogni cambiamento di 
  \textit{leadership} richiede una fase di riconfigurazione che può bloccare temporaneamente il sistema.

  \item \textbf{Tolleranza ai Guasti Parziale}: Sebbene \textit{Raft} sia tollerante ai guasti, può avere difficoltà a gestire partizioni di rete prolungate, note come "\textit{split-brain}", in cui una parte del \textit{cluster}
  è completamente isolata dal resto. In questi casi, potrebbe essere difficile garantire che il \textit{leader} eletto in una partizione abbia ancora la maggioranza.

  \item \textbf{\textit{Overhead} di Replica}: Poiché in \textit{Raft} tutti i nodi devono replicare l'intero \textit{log} delle modifiche, ciò può comportare un significativo \textit{overhead} in termini di memoria e 
  \textit{storage}, specialmente in scenari con grandi volumi di dati.

  \item \textbf{Limitazioni di Latenza}: La latenza di rete nelle applicazioni distribuite geograficamente può aumentare notevolmente i tempi di risposta di \textit{Raft}, poiché ogni decisione deve passare attraverso il 
  \textit{leader} e ottenere il consenso dalla maggioranza dei nodi. Questo può rappresentare un problema per sistemi distribuiti a livello globale.
\end{itemize}





\subsection{Dettagli implementativi dell'applicativo}
\subsubsection{Obiettivi}
% Sezione vuota per scaletta
%TODO: ingloberei questa sotto sezione alla sezione Obiettivi

\subsubsection{Struttura del codice}
% Sezione vuota per scaletta

\subsubsection{Funzionamento}
% Sezione vuota per scaletta

\subsubsection{Limiti}
% Sezione vuota per scaletta
Anche se il nostro sistema funziona come descritto nel documento (citare documento), presenta un grave difetto. La Figura 3 illustra la parte del sistema principalmente coinvolta nel processo di \textit{commit} degli indici.
\\
\begin{lstlisting}[language=Go]
func (c *commonMatchImp) 
    checkUpdateNewMatch(sub *substriber) { 
  var halfNodeNum = c.numNodes/2 
  for c.run { 
    var newMatch = <-sub.Snd 
    if newMatch > c.commonMatchIndex && 
        sub.Trd < newMatch { 
      c.numStable++ 
      if c.numStable > uint(halfNodeNum) {
        if c.commitEntryC != nil {
          c.commitEntryC <- c.commonMatchIndex
        } 
        c.commonMatchIndex++
        c.numStable = 1
      } 
    } 
    sub.Trd = newMatch
  } 
}
\end{lstlisting}
Il problema deriva da una sezione critica del codice a cui accedono contemporaneamente più Goroutine senza alcun meccanismo di controllo dell'accesso. In particolare, è possibile che una Goroutine 
superi il controllo iniziale ed esegua le operazioni previste, e che subito dopo una seconda Goroutine superi lo stesso controllo. A questo punto, però, la prima Goroutine ha già modificato i valori condivisi all'interno del 
corpo dell'istruzione if, il che significa che la seconda Goroutine opera su valori non aggiornati, invalidando di fatto la condizione del guard. Anche se abbiamo eseguito numerosi test (che riconosciamo essere insufficienti 
per dimostrare l'assenza di bug) e abbiamo trovato difficile riprodurre le condizioni necessarie per far emergere questa condizione di race, non abbiamo ancora osservato questo problema nella pratica.
