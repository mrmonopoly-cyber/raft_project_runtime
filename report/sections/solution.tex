\section{Soluzioni}
Per il raggiungimento degli obbiettivi sopra citati sono stati fatti,
al momento della progettazione, degli accorgimenti sui diversi 
elementi che compongono il \textit{cluster}.
Abbiamo deciso di utilizzare \textbf{libvirt} come \textbf{Gestore dei nodi} cosi' da poter astrarre 
dall'\textit{\textbf{hypervisor}} e permettere una maggiore scalabilita' del servizio.
Per la disponibilita' abbiamo deciso di usare il protocollo \textit{\textbf{Raft}} per la
gestione del consenso distribuito. L'omogeneita' dei nodi e' garantita dal protocollo appena 
citato. Per finire sono state anche scritte due \textbf{CLI} (Command Line Interface): 
una di controllo/configurazione e una di utilizzo finale.
La prima server al sistemista per gestire il cluster mentre la seconda serve all'utente finale 
per usufruire delle feature offerte dal cluster.
        
% Sezione vuota per scaletta
