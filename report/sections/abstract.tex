
\section{Preambolo}
Il mondo di oggi è estremamente interconnesso e complesso. Per questa ragione, ogni applicativo pensato per il grande pubblico deve garantire l'affidabilità del servizio in ogni momento. Per risolvere questi problemi %TODO: prima non descrivi un problema, più una caratteristica che devono avere certi sistemi, cambierei la parola problema
, vengono spesso utilizzati sistemi distribuiti in rete, detti anche cluster, che permettono di garantire la scalabilità e l'affidabilità del servizio. Tuttavia, questi sistemi sono estremamente complessi e difficili da installare. Basti pensare alla procedura di configurazione di un cluster Kubernetes o Ceph, che è delicata e richiede molto tempo per essere completata, a meno che non si utilizzino strumenti esterni %TODO: tipo quali strumenti?
per facilitare il lavoro, che non sempre sono affidabili. Tale complessità risiede, tra i tanti fattori, nell'architettura scelta per modellare il sistema, solitamente di tipo Orchestrator. Questo modello offre numerosi vantaggi, ma a un costo: i nodi che compongono il cluster non sono omogenei %TODO: cosa si intende per omogenei
. Questo dettaglio, per quanto sembri insignificante, è ciò che rende complessa la procedura di configurazione, aggiornamento e mantenimento del cluster. In questo paper descriveremo come abbiamo creato un cluster che garantisce le richieste di affidabilità necessarie, permettendo inoltre una veloce e semplice installazione/configurazione.
