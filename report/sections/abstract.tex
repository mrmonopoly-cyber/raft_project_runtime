
\section{Abstract}
The world today is highly interconnected and complex. For this reason, every application 
designed for the general public must ensure reliable service at all times. To address 
these challenges, networked distributed systems, also known as \textit{clusters}, are 
often used. These systems enable scalability and reliability of services. However, they 
are inherently complex and challenging to set up. Consider, for instance, the configuration 
process of a \textit{Kubernetes} or \textit{Ceph} cluster, which is intricate and 
time-consuming unless external tools are used to simplify the task—tools that are not always 
trustworthy.  

This complexity is due, among other factors, to the architecture chosen to model the system, 
typically an \textit{Orchestrator}-based model. While this model offers numerous advantages, 
it comes at a cost: the nodes within the cluster are not homogeneous.  

This seemingly minor detail is, in fact, the root cause of the complexity in configuring, 
updating, and maintaining the cluster.  

In this report, we will describe how we created a cluster that meets the necessary reliability 
requirements while also allowing for quick and straightforward installation and configuration.
