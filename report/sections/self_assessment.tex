
\section{Valutazioni}
\subsection{Raggiungimento obiettivi}
% Sezione vuota per scaletta

\subsection{Difficoltà riscontrate}
Inizialmente, la sfida principale consisteva nella gestione di \textit{libvirt}, in particolare nell'
impostazione delle macchine virtuali (VM) e nella configurazione di sottoreti distinte per separare la rete 
pubblica dalla rete privata. A causa della complessità di \textit{libvirt}, comprendere i suoi meccanismi 
intricati richiedese una ripida curva di apprendimento. Come già accennato, il nostro obiettivo era creare un 
ambiente *snello e manutenibile*, ma raggiungere questo risultato si è rivelato tutt'altro che semplice.

Uno degli aspetti più impegnativi è stato progettare e sviluppare un sistema distribuito. Questo compito è 
notoriamente difficile, infatti sistemi di questo callibro presentano una complessità 
intrinseca e dei requisiti stringenti come, per esempio, la consistenza, la resistenza ai guasti e la 
coordinazione tra nodi. A questo si aggiunge \textit{Raft}:
tradurre i suoi principi e le sue linee guida presentate nell'articolo (citare articolo) in una soluzione 
concreta ha richiesto una quantità di tempo considerevole ed è stata, simultaneamente, una sfida impegnativa da 
un punto di vista mentale.
Inoltre, puntavamo a sviluppare un sistema scalabile, con diversi livelli di astrazione e composto da 
componenti scarsamente accoppiati tra loro. Questo era essenziale poiché, 
con la crescita del progetto, la gestione del codice diventava sempre più difficile, portando a un crescente 
debito tecnico. Di conseguenza, abbiamo effettuato diverse revisioni del design architettonico per affrontare 
questi problemi.

Nonostante i nostri sforzi per limitare il debito tecnico, esso è rimasto una sfida costante. La base del codice 
diventava vasta e complessa, il che complicava la gestione delle \textit{goroutine} e di tutti gli aspetti 
relativi alla sincronizzazione tra di esse, e persino il semplice compito di seguire il flusso del sistema. 
Anche i test si sono rivelati difficili a causa di questa complessità. Abbiamo trovato particolarmente 
complicato progettare il sistema di configurazione, poiché non fu immediatamente chiaro come inserirlo nel 
sistema già presente. Abbiamo impiegato tempo e sforzi, sia nella rilettura dell'articolo di riferimento, sia nel
la ristrutturazione del codice, per arrivare ad una soluzione corretta e funzionante

Un altro ostacolo significativo è stato eseguire le operazioni di \textit{testing} sul sistema. Gestire più 
macchine virtuali ed effettuare \textit{debug} tramite semplici stampe  
si è rivelato macchinoso e dispendioso in termini di tempo. Un sistema di \textit{logging} distribuito sarebbe 
stato una 
soluzione più efficace, ma implementare un tale sistema avrebbe richiesto un progetto aggiuntivo.

In retrospettiva, bilanciare la complessità tecnica mantenendo al contempo modularità, scalabilità e 
un'architettura pulita è stata la sfida più grande che abbiamo affrontato durante tutto il progetto.
\subsubsection{Implementazione del protocollo Raft}
% Sezione vuota per scaletta
\subsubsection{Modularizzazione}
% Sezione vuota per scaletta



\subsubsection{Organizzazione dei test}
% Sezione vuota per scaletta

\subsubsection{Raccolta dei log}
% Sezione vuota per scaletta

\subsubsection{Automatizzare il processo di deployment del cluster in produzione}
% Sezione vuota per scaletta

\subsubsection{Comprensione del cambio di configurazione del protocollo Raft}
% Sezione vuota per scaletta

\subsubsection{Riconoscimento dei casi di race condition}
% Sezione vuota per scaletta

\subsection{Vincoli e miglioramenti realizzabili}
% Sezione vuota per scaletta
