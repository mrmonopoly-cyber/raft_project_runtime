\subsubsection{File System abstraction}
Il cluster è progettato con l'obiettivo principale di fornire un file system distribuito. In questo contesto, abbiamo ritenuto 
indispensabile definire un livello di astrazione del file system per ogni nodo, al fine di facilitare l'interazione con esso e 
garantire un'implementazione più modulare e gestibile.\\
L'astrazione del file system di ciascun nodo rappresenta un'interfaccia standardizzata che consente di eseguire operazioni sui 
file in modo uniforme, indipendentemente dalle specificità del nodo. Questo approccio consente di mantenere
un elevato grado di coerenza e semplicità nella gestione del file system distribuito.
Il cuore dell'astrazione risiede in un singolo metodo principale, chiamato ApplyLogEntry, che svolge un ruolo centrale nell'elaborazione
delle operazioni sui file. Questo metodo è progettato per accettare una "log entry" come input e 
gestire le seguenti funzioni: il metodo analizza la log entry per identificare il tipo di operazione da eseguire: creazione 
di un nuovo file, scrittura su un file, rinomina di un file, eliminazione di un file; in seguito, procede con 
l'esecuzione dell'operazione corrispondente sul file system del nodo.

\subsubsection{Instaurazione connessione tra nodi}
All'avvio, il nodo lancia diverse goroutine: innanzitutto, si mette in ascolto di nuove connessioni su una porta 
e comincia a connettersi ai nodi della configurazione iniziale
(ciascun nodo parte con una configurazione iniziale, questo per evitare la complessità di costruire una configurazione iniziale strada 
facendo). Dopodichè rimane in ascolto di nuove connessioni e ogni volta che ne arriva una nuova, il nodo controlla
l'indirizzo IP, lo distingue tra indirizzo "interno" ed "esterno" e lancia una goroutine per ciascuna di queste
categorie e, infine, aspetta di ricevere messaggi da queste connessioni. \\
Per l'instaurazione delle connesioni, il nostro sistema fa uso di un modulo apposito, chiamato \texttt{Node}, che 
rappresenta un singolo nodo e la sua connessione di riferimenot, ed espone i metodi per comunicare 
con esso (invio e ricezione) e gestire la connesione (chiusura e aggiunta).
