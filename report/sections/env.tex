\section{Configurazione dell'ambiente}
Per poter utilizzare correttamente il cluster e' necessario prima di tutto configurare
l'ambienete che si occupera' dell'instanziazione dello stesso. In particolare sara' necessario:
\begin{itemize}
    \item installare libvirt aggiungendo i corretti all'utente e altri applicativi
    \item creare l'immagine di sistema che verra' utilizzata al boot delle VM
\end{itemize}

Per automatizzare il processo e' stato scritto uno script bash che permetta, 
su distribuzioni basate su arch-linux, di configurare tutto l'occorrente.
Il motivo di tale restrizione e' che per creare l'immagine viene usata uno strumento
esclusivo di arch-linux chiamato Archiso \url{https://wiki.archlinux.org/title/Archiso}.

Tale strumento permette di creare immagini di sistema ed e' tipicamente utlizzato per creare 
Live di installazione personalizzate. 

Usando questo strumento abbiamo organizzato il filesystem, i servizi, i pacchetti installati e il 
bootloader.

Il nostro obbiettivo e' stato quello di minimizzare la dimensione dell'immagine rimuovendo programmi
unitili e\/o sostituendoli con alternative piu' leggere.
Abbiamo inoltre scelto di usare come bootloader \textbf{syslinux} in favore della sua leggerezza 
rispetto allo standard \textbf{GRUB} impostando un timeout di avvio uguale a zero cosi' da diminuire
i tempi di avvio.

Una volta creata l'immagine viene copiata nelle cartelle di default di \textbf{Libvirt} 
(\/var\/lib\/libvirt\/images\/) cosi' che libvirt possa utilizzarla nella fase di creazione
delle VM.

All'avvio dello script bash quello che quindi accade e' che vengono installati i pacchetti
necessari per gestire la virtualizzazione, vengono forniti i permessi necessaru all'attuale utente
per utilizzare Libvirt senza essere l'utente \textbf{ROOT} e viene poi utilizzato 
\textbf{archiso} per creare l'immagine di sistema.

All'inizio avevamo pensato di usare una immagine precompilata da poi fornire ma abbiamo preferito
utilizzare un sistema come quello citato poiche' e' molto piu' sicuro e non ci richiede di tenere
traccia manualmente delle versioni del sistema. Inoltre fornisce una semplice procedura
per l'aggiornamento del cluster, in particolare e' sufficente modificare la configurazione di 
archiso e rieseguire lo script di configurazione per installare gli aggiornamenti apportati. 
