\section{Environment configuration}
To use the cluster correctly, it is first necessary to configure the 
environment responsible for its instantiation. Specifically, the following
steps are required:
\begin{itemize}
	\item Install \textit{libvirt}, ensuring the user has the necessary permissions,
    along with other required applications.
	\item Create the system image that will be used to boot the virtual
    machines (VMs).
\end{itemize}
To automate this process, we developed a Bash script that configures 
everything needed on Arch Linux-based distributions.

The reason for this restriction is that the image creation process uses 
an Arch Linux-exclusive tool called \textit{Archiso}\cite{6}. 
This tool allows the creation of system images and is typically used 
to generate customized installation Live environments.

Using \textit{Archiso}, we organized the filesystem, configured services, selected 
installed packages, and set up the bootloader.

Our objective was to minimize the image size by removing unnecessary programs 
and/or replacing them with lighter alternatives. We also chose \textit{Syslinux} 
as the bootloader due to its lightweight nature compared to the standard 
\textit{GRUB}, setting a boot timeout of zero to reduce startup times.

Once the image is created, it is copied into the default 	\textit{libvirt} directories
(\texttt{/var/lib/libvirt/images/}) so that \textit{libvirt} can use it during 
the VM creation process.

When the Bash script is executed, it performs the following steps:
\begin{itemize}
	\item Installs the required packages for virtualization.
	\item Grants the current user the necessary permissions to use \textit{libvirt} 
    without requiring root privileges.
  \item Uses \textit{Archiso} to create the system image.
\end{itemize}
      Initially, we considered using a precompiled image, but we opted for this 
      approach because it is significantly more secure and eliminates the need to 
      manually track system versioning. Additionally, it provides a simple update 
      process for the cluster. Specifically, updating the cluster only requires modifying 
      the \textit{Archiso} configuration and re-executing the setup script to apply 
      the desired updates.
