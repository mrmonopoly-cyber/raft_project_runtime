\documentclass[acmtog]{acmart}

\AtBeginDocument{%
  \providecommand\BibTeX{{%
    Bib\TeX}}}

%%
%% These commands are for a JOURNAL article.

%\citestyle{acmauthoryear}
\usepackage[italian]{babel}
\usepackage[T1]{fontenc}
\usepackage{amsmath}
\usepackage{mathtools}
\usepackage{amssymb}
\usepackage{stmaryrd}

\title{PIB: Una soluzione alla complessità di installazione dei cluster moderni}
%%%%%%%%%%%%%%%%%%%%%%%%%%%%%%%%%%%%%%%%%%%%%%%%%%%%%%%%%%%%%%%%%%%%%%%%%%%%%%%%%%%%%%%%%%%%%%%%%%%%%%%%%%%%%%%%%%%%%%%%%%%%%%
%%%%%%%%%%%%%%%%%%%%%%%%%%%%%%%%%%%%%%%%%%%%%%%%%%%%%%%%%%%%%%%%%%%%%%%%%%%%%%%%%%%%%%%%%%%%%%%%%%%%%%%%%%%%%%%%%%%%%%%%%%%%%%
\author{Alberto Damo}
\authornote{Entrambi gli autori hanno contribuito equamente a questo progetto.}
\email{alberto.damo@studenti.unipd.it}
\author{Alessandro Pirolo}
\authornotemark[1]
\email{alessandro.pirolo@studenti.unipd.it}
\affiliation{%
  \institution{Università degli Studi di Padova}
  \city{Padova}
  \state{Veneto}
  \country{Italia}
}

%%
%% By default, the full list of authors will be used in the page
%% headers. Often, this list is too long, and will overlap
%% other information printed in the page headers. This command allows
%% the author to define a more concise list
%% of authors' names for this purpose.
\renewcommand{\shortauthors}{Damo, Pirolo}

%%
%% The abstract is a short summary of the work to be presented in the
%% article.
\begin{abstract}
  A %TODO: Sposterei il preambolo qui
\end{abstract}


%%%%%%%%%%%%%%%%%%%%%%%%%%%%%%%%%%%%%%%%%%%%%%%%%%%%%%%%%%%%%%%%%%%%%%%%%%%%%%%%%%%%%%%%%%%%%%%%%%%%%%%%%%%%%%%%%%%%%%%%%%%%%%
%%%%%%%%%%%%%%%%%%%%%%%%%%%%%%%%%%%%%%%%%%%%%%%%%%%%%%%%%%%%%%%%%%%%%%%%%%%%%%%%%%%%%%%%%%%%%%%%%%%%%%%%%%%%%%%%%%%%%%%%%%%%%%
\begin{document}

\maketitle


\section{Introduction}
The world today is highly interconnected and complex. For this reason, every application 
designed for the general public must ensure reliable service at all times. To address 
these challenges, networked distributed systems, also known as \textit{clusters}, are 
often used. These systems enable scalability and reliability of services. However, they 
are inherently complex and challenging to set up. Consider, for instance, the configuration 
process of a \textit{Kubernetes} or \textit{Ceph} cluster, which is intricate and 
time-consuming unless external tools are used to simplify the task—tools that are not always 
trustworthy.  

This complexity is due, among other factors, to the architecture chosen to model the system, 
typically an \textit{Orchestrator}-based model. While this model offers numerous advantages, 
it comes at a cost: the nodes within the cluster are not homogeneous.  

This seemingly minor detail is, in fact, the root cause of the complexity in configuring, 
updating, and maintaining the cluster.  

In this report, we will describe how we created a cluster that meets the necessary reliability 
requirements while also allowing for quick and straightforward installation and configuration.


\section{Composizione di un cluster}
Prima di poter analizzare i nostri obiettivi, come li abbiamo raggiunti e perché siano state fatte determinate scelte, è necessario descrivere gli elementi che compongono un \textit{cluster} e le loro caratteristiche:

\begin{itemize}
  \item \textbf{Nodi}: Sono i computer che si occupano della computazione necessaria per mantenere il servizio attivo. Possono essere fisici (\textit{server} dedicati) o virtuali (macchine virtuali, \textit{container}).
  \item \textbf{Gestore dei nodi}: Questo componente si occupa dell'aggiunta, rimozione e sostituzione dei nodi nel \textit{cluster}. Può essere un \textit{hypervisor}, nel caso i nodi siano virtuali, o un operatore, nel caso i nodi siano server reali. È probabile che ci siano più gestori, soprattutto se i nodi sono geograficamente distanti.
  \item \textbf{Gestore di rete}: Questo componente gestisce le connessioni interne ed esterne al \textit{cluster}. In particolare, si occupa dell'assegnazione degli indirizzi di rete ai nodi (è possibile che ogni nodo abbia più interfacce di rete e quindi più indirizzi IP).
  \item \textbf{Sistema operativo}: Sebbene appartenga al nodo, il sistema operativo deve essere progettato su misura per il \textit{cluster}, poiché anche questo componente è responsabile per la corretta operatività dell'intera infrastruttura. In particolare, dovrà:
    \begin{itemize}
        \item Ottenere gli indirizzi di rete (è possibile siano più di uno per ogni nodo).
        \item Eseguire una diagnostica interna sullo stato della macchina.
        \item Eseguire eventuali programmi o daemon responsabili delle funzionalità del \textit{cluster} e degli applicativi.
    \end{itemize}
\end{itemize}


\section{Obiettivi}
Come già accennato, il nostro obiettivo è realizzare un \textit{cluster} che sia allo stesso tempo affidabile e semplice da configurare/installare.

In particolare, il sistema dovrà:
\begin{itemize}
    \item Permettere la scalabilità del servizio aumentando o diminuendo il numero di nodi senza interrompere i servizi forniti.
    \item Garantire la disponibilità del servizio anche in caso di guasti sia a livello fisico (\textit{server}, rete), sia a livello \textit{software} (\textit{crash} del sistema operativo o dell'applicativo stesso).
    \item Integrare tool per l'interazione dell'amministratore con il cluster.
    \item Essere privo di \textit{single point of failure}.
    \item Avere tutti i nodi del \textit{cluster} uguali tra di loro a livello strutturale.
\end{itemize}
% TODO: cambierei alcuni nomi dall'inglese all'italiano, quelli che rimangono in inglese vanno messi in conrsivo

\section{Soluzioni}
Per il raggiungimento degli obbiettivi sopra citati sono stati fatti,
al momento della progettazione, degli accorgimenti sui diversi 
elementi che compongono il \textit{cluster}.
Abbiamo deciso di utilizzare \textbf{libvirt} come \textbf{Gestore dei nodi} cosi' da poter astrarre 
dall'\textit{\textbf{hypervisor}} e permettere una maggiore scalabilita' del servizio.
Per la disponibilita' abbiamo deciso di usare il protocollo \textit{\textbf{Raft}} per la
gestione del consenso distribuito. L'omogeneita' dei nodi e' garantita dal protocollo appena 
citato. Per finire sono state anche scritte due \textbf{CLI} (Command Line Interface): 
una di controllo/configurazione e una di utilizzo finale.
La prima server al sistemista per gestire il cluster mentre la seconda serve all'utente finale 
per usufruire delle feature offerte dal cluster.
        
% Sezione vuota per scaletta


\section{PIB}
\subsection{Elementi}
\subsubsection{Nodi}
% Sezione vuota per scaletta

\subsubsection{Gestore dei nodi e della rete}
% Sezione vuota per scaletta
Abbiamo progettato l'architettura di rete del \textit{cluste}r con due \textit{subnet} distinte, ciascuna con ruoli specifici per garantire la separazione tra il traffico pubblico rivolto ai \textit{client} e la comunicazione interna del \textit{cluster}:
\begin{itemize}
  \item \textit{Subnet} Pubblica (192.168.2.0/24): Questa \textit{subnet} è basata su \textit{NAT}, il che significa che i nodi nella rete pubblica possono accedere a \textit{internet} esterno e comunicare con sistemi esterni, 
    ma non possono comunicare direttamente tra loro all'interno della \textit{subnet}. Il \textit{pool} di rete pubblica è riservato ai nodi che devono interagire con servizi o client esterni. In questo intervallo, 
    gli indirizzi 1 e 255 sono riservati rispettivamente all'\textit{hypervisor} e ai messaggi di \textit{broadcast}, mentre i restanti indirizzi IP sono assegnati dinamicamente alle macchine virtuali che necessitano di 
    accesso esterno.
  \item \textit{Subnet} Privata (10.0.0.0/24): Questa \textit{subnet} è utilizzata esclusivamente per la comunicazione \textit{intra-cluster}. L'abbiamo creata utilizzando un \textit{bridge} virtuale, che consente a tutti i 
    nodi della \textit{subnet} di comunicare direttamente tra loro. Tuttavia, questa rete non ha accesso a \textit{internet} esterno, assicurando che il suo unico scopo sia facilitare la comunicazione tra i nodi per compiti 
    correlati al \textit{cluster}. Come nella \textit{subnet} pubblica, gli indirizzi 1 e 255 sono riservati, e il resto dell'intervallo è disponibile per i nodi del \textit{cluster}.
\end{itemize}

Separando le \textit{subnet} pubbliche e private, garantiamo che le richieste dei \textit{client} e i messaggi \textit{intra-cluster} siano instradati correttamente senza interferenze. La \textit{subnet} pubblica gestisce 
le interazioni con i \textit{client}, mentre la \textit{subnet} privata è dedicata alle operazioni interne, come il coordinamento e la replica dello stato tra i nodi, come richiesto dal protocollo \textit{Raft}. Questa separazione 
non solo migliora la sicurezza e le prestazioni, ma facilita anche la manutenibilità e la scalabilità del \textit{cluster}.

\subsubsection{Sistema operativo}
% Sezione vuota per scaletta
Nel nostro \textit{cluster}, ogni macchina virtuale esegue \textit{Arch Linux} come sistema operativo. All'avvio, due processi chiave vengono avviati automaticamente su ogni nodo.
\begin{itemize}
  \item \textit{\textbf{raft\_daemon.service}}: Questo processo è responsabile dell'esecuzione del codice relativo al protocollo di consenso \textit{Raft}, che garantisce che il nostro sistema distribuito mantenga la tolleranza 
    ai guasti e la consistenza tra tutti i nodi.
  \item \textit{\textbf{discovery.service}}: Questo processo facilita la scoperta degli indirizzi IP degli altri nodi all'interno della \textit{subnet} 10.0.0.x/24. Lo fa eseguendo scansioni di rete periodiche utilizzando 
    \textit{nmap}. Ogni 30 secondi, il processo di \textit{discovery} recupera e memorizza gli indirizzi IP delle interfacce di rete pubbliche e private degli altri nodi. Questa ricerca continua garantisce che ogni nodo 
    rimanga "consapevole" degli altri nodi all'interno della \textit{subnet}, mantenendo così una comunicazione continua tra loro.
\end{itemize}



\subsection{Raft per il consenso distribuito}
\textit{Raft} è un algoritmo di consenso distribuito, ovvero un meccanismo che permette a un gruppo di computer (nodi) di raggiungere un accordo su un dato stato, anche in presenza di guasti o latenze di rete. In pratica, 
\textit{Raft} assicura che tutti i nodi abbiano una copia identica e aggiornata dei dati, garantendo così la coerenza e l'affidabilità di un sistema distribuito.

\subsubsection{Motivazioni}
% Sezione vuota per scaletta
Viene spesso preferito per il consenso distribuito per diverse ragioni:
\begin{itemize}
  \item \textbf{Semplicità}: \textit{Raft} è stato progettato per essere più comprensibile rispetto ad altri algoritmi di consenso. Un esempio è \textit{Paxos}, che ha acquisito una reputazione per la sua difficoltà sia 
  nell'implementazione che nella comprensione. È proprio questa semplicità che lo rende accessibile agli sviluppatori.
  
  \item \textbf{Tolleranza ai Guasti}: In ambienti distribuiti, è molto probabile che uno o più nodi si guastino. \textit{Raft}, tramite i suoi meccanismi, può garantire che il sistema continuerà a funzionare correttamente 
  anche se alcuni nodi non sono attivi o non rispondono.

  \item \textbf{Replica Uniforme dei Dati}: \textit{Raft} assicura che i dati siano replicati uniformemente tra i nodi, in modo che essi mantengano copie esatte dello stato condiviso, il che assicura alta disponibilità e resilienza.

  \item \textbf{Leadership Chiaramente Definita}: A differenza di \textit{Paxos}, \textit{Raft} semplifica la gestione del consenso eleggendo un \textit{leader} tra i nodi. Il \textit{leader} organizza la replica dei dati 
  e coordina le decisioni critiche, riducendo la complessità dell'algoritmo.

  \item \textbf{Efficienza}: \textit{Raft} può raggiungere il consenso in tempi relativamente brevi; pertanto, è adatto per applicazioni che richiedono velocità e bassa latenza, riducendo i ritardi nella replica dei dati.

\end{itemize}


\subsubsection{Applicazione}
% Sezione vuota per scaletta
Raft si applica a una vasta gamma di applicazioni distribuite, specialmente in scenari che richiedono alta disponibilità, tolleranza ai guasti e coerenza dei dati. Esempi di applicazioni includono:

\begin{itemize}
  \item \textbf{Database Distribuiti}: \textit{Raft} è utilizzato, come in \textit{Etcd} e \textit{TiKV}, per replicare i dati in modo consistente su più nodi, garantendo che, anche in caso di guasti \textit{hardware} o di rete, 
  nessun dato venga perso.

 \item \textbf{Sistemi di File Distribuiti}: Per sistemi che distribuiscono \textit{file} su vari nodi, come \textit{HDFS}, \textit{Raft} garantisce che le modifiche ai \textit{file} siano applicate in modo coerente mantenendo 
 il sistema disponibile durante i guasti dei nodi.

 \item \textbf{Sistemi di Coordinamento e \textit{Registry}}: Il protocollo \textit{Raft} è utilizzato in servizi come \textit{Consul} ed \textit{Etcd} per la configurazione distribuita e la scoperta dei servizi. Permette a un 
 \textit{cluster} di nodi di raggiungere il consenso sui cambiamenti di stato in modo affidabile.

 \item \textbf{\textit{Blockchain} Private}: Alcune implementazioni di \textit{blockchain} private, come \textit{Hyperledger}, utilizzano \textit{Raft} poiché raggiungere rapidamente il consenso con tolleranza ai guasti per 
 la validazione delle transazioni è fondamentale.

 \item \textbf{\textit{Cache} Distribuite}: Nei sistemi di \textit{caching} distribuito come \textit{Redis} in modalità \textit{cluster}, \textit{Raft} può essere utilizzato per garantire che gli aggiornamenti 
 effettuati alla \textit{cache} siano propagati in modo coerente tra i nodi del \textit{cluster}.
\end{itemize}

\subsubsection{Limiti}
% Sezione vuota per scaletta
Sebbene Raft presenti molti vantaggi, ci sono alcune limitazioni da considerare durante il processo di progettazione e implementazione:
\begin{itemize}
  \item \textbf{Scalabilità Limitata}: Sebbene \textit{Raft} sia molto più semplice rispetto agli altri algoritmi di consenso, soffre di problemi di scalabilità in grandi \textit{cluster}. Un \textit{leader} centralizzato limita 
  la capacità di scrittura alla velocità di elaborazione del \textit{leader} stesso. Questo può diventare un collo di bottiglia in ambienti con centinaia o migliaia di nodi.

  \item \textbf{Dipendenza dal \textit{Leader}}: Il meccanismo di elezione del \textit{leader} può risultare problematico se il \textit{leader} cambia frequentemente, come in ambienti instabili. Ogni cambiamento di 
  \textit{leadership} richiede una fase di riconfigurazione che può bloccare temporaneamente il sistema.

  \item \textbf{Tolleranza ai Guasti Parziale}: Sebbene \textit{Raft} sia tollerante ai guasti, può avere difficoltà a gestire partizioni di rete prolungate, note come "\textit{split-brain}", in cui una parte del \textit{cluster}
  è completamente isolata dal resto. In questi casi, potrebbe essere difficile garantire che il \textit{leader} eletto in una partizione abbia ancora la maggioranza.

  \item \textbf{\textit{Overhead} di Replica}: Poiché in \textit{Raft} tutti i nodi devono replicare l'intero \textit{log} delle modifiche, ciò può comportare un significativo \textit{overhead} in termini di memoria e 
  \textit{storage}, specialmente in scenari con grandi volumi di dati.

  \item \textbf{Limitazioni di Latenza}: La latenza di rete nelle applicazioni distribuite geograficamente può aumentare notevolmente i tempi di risposta di \textit{Raft}, poiché ogni decisione deve passare attraverso il 
  \textit{leader} e ottenere il consenso dalla maggioranza dei nodi. Questo può rappresentare un problema per sistemi distribuiti a livello globale.
\end{itemize}





\subsection{Dettagli implementativi dell'applicativo}
\subsubsection{Obiettivi}
% Sezione vuota per scaletta
%TODO: ingloberei questa sotto sezione alla sezione Obiettivi

\subsubsection{Struttura del codice}
% Sezione vuota per scaletta

\subsubsection{Funzionamento}
% Sezione vuota per scaletta

\subsubsection{Limiti}
% Sezione vuota per scaletta

%TODO: sposterei questa sezione a dopo o prima della sezione Soluzioni (dipende da cos'è Soluzione)
\section{Tool di configurazione}
\subsection{Obiettivi}
% Sezione vuota per scaletta

\subsection{Utilizzo}
% Sezione vuota per scaletta

\subsection{Limiti}
% Sezione vuota per scaletta


\section{Valutazioni}
\subsection{Raggiungimento obiettivi}
% Sezione vuota per scaletta
Il progetto ha centrato con successo gli obiettivi principali definiti inizialmente, ponendo le basi per un sistema 
cluster affidabile, scalabile e di facile configurazione. Tra i risultati più rilevanti si evidenzia:
\begin{enumerate}
  \item Affidabilità e Tolleranza ai Guasti: L'adozione del protocollo Raft ha permesso di garantire 
  un consenso distribuito tra i nodi, eliminando qualsiasi single point of failure. Questo approccio assicura 
  che, anche in caso di guasti a livello fisico (server, rete) o software (crash del sistema operativo), 
  il cluster continui a funzionare in maniera coerente, mantenendo l'integrità dello stato condiviso.
  
  \item Scalabilità Dinamica: Il sistema consente l'aggiunta o la rimozione di nodi senza interrompere
  i servizi forniti. L'uso di libvirt come gestore dei nodi ha reso possibile l’astrazione dell’hypervisor,
  semplificando la gestione delle macchine virtuali e supportando facilmente la scalabilità del cluster.

  \item Configurazione Automatizzata: La creazione di script bash ha semplificato e automatizzato la fase 
  di setup dell’ambiente e delle immagini di sistema. Questo approccio riduce significativamente 
  i tempi di configurazione e garantisce consistenza nelle installazioni.

  \item Strumenti di Interazione: Lo sviluppo di tool CLI ha introdotto un'interfaccia intuitiva per 
  amministratori e utenti finali. La CLI consente di monitorare lo stato del cluster, gestire 
  configurazioni dinamiche e inviare richieste specifiche, come il recupero o la modifica di file.

  \item Architettura Modulare: La progettazione del sistema è stata effettuata con un focus sulla modularità
  e sull’isolamento delle componenti. Moduli dedicati, come il raft\_log, ConfPool e RPC module, hanno permesso
  di ottenere un sistema organizzato e manutenibile, facilitando future estensioni e miglioramenti.

  \item Ottimizzazione delle Risorse: La scelta di Arch Linux come sistema operativo ha permesso di minimizzare l’uso delle 
  risorse, grazie a un’immagine di sistema snella e personalizzata con componenti leggeri come Syslinux al posto di GRUB.
\end{enumerate}

\subsection{Difficoltà riscontrate}
Inizialmente, la sfida principale consisteva nella gestione di \textit{libvirt}, in particolare nell'
impostazione delle macchine virtuali (VM) e nella configurazione di sottoreti distinte per separare la rete 
pubblica dalla rete privata. A causa della complessità di \textit{libvirt}, comprendere i suoi meccanismi 
intricati richiedese una ripida curva di apprendimento. Come già accennato, il nostro obiettivo era creare un 
ambiente *snello e manutenibile*, ma raggiungere questo risultato si è rivelato tutt'altro che semplice.

Uno degli aspetti più impegnativi è stato progettare e sviluppare un sistema distribuito. Questo compito è 
notoriamente difficile, infatti sistemi di questo callibro presentano una complessità 
intrinseca e dei requisiti stringenti come, per esempio, la consistenza, la resistenza ai guasti e la 
coordinazione tra nodi. A questo si aggiunge \textit{Raft}:
tradurre i suoi principi e le sue linee guida presentate nell'articolo (citare articolo) in una soluzione 
concreta ha richiesto una quantità di tempo considerevole ed è stata, simultaneamente, una sfida impegnativa da 
un punto di vista mentale.
Inoltre, puntavamo a sviluppare un sistema scalabile, con diversi livelli di astrazione e composto da 
componenti scarsamente accoppiati tra loro. Questo era essenziale poiché, 
con la crescita del progetto, la gestione del codice diventava sempre più difficile, portando a un crescente 
debito tecnico. Di conseguenza, abbiamo effettuato diverse revisioni del design architettonico per affrontare 
questi problemi.

Nonostante i nostri sforzi per limitare il debito tecnico, esso è rimasto una sfida costante. La base del codice 
diventava vasta e complessa, il che complicava la gestione delle \textit{goroutine} e di tutti gli aspetti 
relativi alla sincronizzazione tra di esse, e persino il semplice compito di seguire il flusso del sistema. 
Anche i test si sono rivelati difficili a causa di questa complessità. Abbiamo trovato particolarmente 
complicato progettare il sistema di configurazione, poiché non fu immediatamente chiaro come inserirlo nel 
sistema già presente. Abbiamo impiegato tempo e sforzi, sia nella rilettura dell'articolo di riferimento, sia nel
la ristrutturazione del codice, per arrivare ad una soluzione corretta e funzionante

Un altro ostacolo significativo è stato eseguire le operazioni di \textit{testing} sul sistema. Gestire più 
macchine virtuali ed effettuare \textit{debug} tramite semplici stampe  
si è rivelato macchinoso e dispendioso in termini di tempo. Un sistema di \textit{logging} distribuito sarebbe 
stato una 
soluzione più efficace, ma implementare un tale sistema avrebbe richiesto un progetto aggiuntivo.

In retrospettiva, bilanciare la complessità tecnica mantenendo al contempo modularità, scalabilità e 
un'architettura pulita è stata la sfida più grande che abbiamo affrontato durante tutto il progetto.

\subsection{Miglioramenti realizzabili}
Nonostante i risultati significativi, il progetto presenta alcune criticità e aree di miglioramento che 
potrebbero essere affrontate per rendere il sistema più robusto, efficiente e accessibile.
\begin{enumerate}
  \item Gestione dei Log Distribuiti: Attualmente, il debugging del sistema e la raccolta dei 
  log risultano macchinosi, in quanto basati su semplici stampe. L’implementazione di un sistema 
  di logging distribuito permetterebbe di centralizzare la raccolta dei log, facilitando il 
  monitoraggio delle macchine virtuali e l’identificazione dei problemi in tempo reale. 
  
  \item Risoluzione della Data Race: È stata identificata una possibile data race che povocherebbe 
  una race condition all'interno della procedura di commit dei log, dove più goroutine possono accedere 
  contemporaneamente a risorse condivise senza un meccanismo di controllo adeguato. Per 
  risolvere questo problema, si potrebbe implementare una coda di priorità a doppia struttura,
  garantendo così l'ordine di esecuzione delle operazioni senza 
  introdurre ulteriori overhead o spinlock.

  \item Ottimizzazione delle Performance: Sebbene il sistema funzioni correttamente, 
  alcune operazioni del protocollo Raft potrebbero essere ottimizzate ulteriormente. 
  Ad esempio, la compressione dei log non è attualmente supportata, il che può portare 
  a un aumento delle risorse utilizzate nel tempo. Introdurre una strategia di snapshot
  o di compressione permetterebbe di ridurre lo spazio occupato dai log e migliorare 
  le performance complessive.

  \item Automazione del Deployment in Produzione: Anche se è stata realizzata 
  un’automazione per la configurazione iniziale, il deployment del cluster in ambienti 
  di produzione richiede ancora un certo intervento manuale. 

  \item Interfaccia Grafica (GUI): L'uso esclusivo della CLI per interagire con il 
  cluster potrebbe risultare poco accessibile per utenti meno esperti o non tecnici. 
  Lo sviluppo di un’interfaccia grafica semplificata permetterebbe di visualizzare lo stato 
  dei nodi, monitorare le metriche del sistema e gestire le configurazioni con maggiore intuitività.

  \item Testing e Validazione del Sistema: La complessità del sistema ha reso difficoltosa 
  l’esecuzione dei test e la validazione delle funzionalità. Potrebbe essere utile introdurre 
  una suite di test automatizzati, anche se non sarebbero sufficienti a garantire un 
  comportamento corretto.
\end{enumerate}


\end{document}
