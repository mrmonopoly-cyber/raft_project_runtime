\documentclass[acmtog]{acmart}

\AtBeginDocument{%
  \providecommand\BibTeX{{%
    Bib\TeX}}}

%%
%% These commands are for a JOURNAL article.

%\citestyle{acmauthoryear}
\usepackage[italian]{babel}
\usepackage[T1]{fontenc}
\usepackage{amsmath}
\usepackage{mathtools}
\usepackage{amssymb}
\usepackage{stmaryrd}
\usepackage{listings}

\title{PIB: Una soluzione alla complessità di installazione dei cluster moderni}
%%%%%%%%%%%%%%%%%%%%%%%%%%%%%%%%%%%%%%%%%%%%%%%%%%%%%%%%%%%%%%%%%%%%%%%%%%%%%%%%%%%%%%%%%%%%%%%%%%%%%%%%%%%%%%%%%%%%%%%%%%%%%%
%%%%%%%%%%%%%%%%%%%%%%%%%%%%%%%%%%%%%%%%%%%%%%%%%%%%%%%%%%%%%%%%%%%%%%%%%%%%%%%%%%%%%%%%%%%%%%%%%%%%%%%%%%%%%%%%%%%%%%%%%%%%%%
\author{Alberto Damo}
\authornote{Entrambi gli autori hanno contribuito equamente a questo progetto.}
\email{alberto.damo@studenti.unipd.it}
\author{Alessandro Pirolo}
\authornotemark[1]
\email{alessandro.pirolo@studenti.unipd.it}
\affiliation{%
  \institution{Università degli Studi di Padova}
  \city{Padova}
  \state{Veneto}
  \country{Italia}
}

%%
%% By default, the full list of authors will be used in the page
%% headers. Often, this list is too long, and will overlap
%% other information printed in the page headers. This command allows
%% the author to define a more concise list
%% of authors' names for this purpose.
\renewcommand{\shortauthors}{Damo, Pirolo}

%%
%% The abstract is a short summary of the work to be presented in the
%% article.
\begin{abstract}
  A %TODO: Sposterei il preambolo qui
\end{abstract}


%%%%%%%%%%%%%%%%%%%%%%%%%%%%%%%%%%%%%%%%%%%%%%%%%%%%%%%%%%%%%%%%%%%%%%%%%%%%%%%%%%%%%%%%%%%%%%%%%%%%%%%%%%%%%%%%%%%%%%%%%%%%%%
%%%%%%%%%%%%%%%%%%%%%%%%%%%%%%%%%%%%%%%%%%%%%%%%%%%%%%%%%%%%%%%%%%%%%%%%%%%%%%%%%%%%%%%%%%%%%%%%%%%%%%%%%%%%%%%%%%%%%%%%%%%%%%
\begin{document}

\maketitle


\section{Introduction}
The world today is highly interconnected and complex. For this reason, every application 
designed for the general public must ensure reliable service at all times. To address 
these challenges, networked distributed systems, also known as \textit{clusters}, are 
often used. These systems enable scalability and reliability of services. However, they 
are inherently complex and challenging to set up. Consider, for instance, the configuration 
process of a \textit{Kubernetes} or \textit{Ceph} cluster, which is intricate and 
time-consuming unless external tools are used to simplify the task—tools that are not always 
trustworthy.  

This complexity is due, among other factors, to the architecture chosen to model the system, 
typically an \textit{Orchestrator}-based model. While this model offers numerous advantages, 
it comes at a cost: the nodes within the cluster are not homogeneous.  

This seemingly minor detail is, in fact, the root cause of the complexity in configuring, 
updating, and maintaining the cluster.  

In this report, we will describe how we created a cluster that meets the necessary reliability 
requirements while also allowing for quick and straightforward installation and configuration.


\section{Composizione di un cluster}
Prima di poter analizzare i nostri obiettivi, come li abbiamo raggiunti e perché siano state fatte determinate scelte, è necessario descrivere gli elementi che compongono un \textit{cluster} e le loro caratteristiche:

\begin{itemize}
  \item \textbf{Nodi}: Sono i computer che si occupano della computazione necessaria per mantenere il servizio attivo. Possono essere fisici (\textit{server} dedicati) o virtuali (macchine virtuali, \textit{container}).
  \item \textbf{Gestore dei nodi}: Questo componente si occupa dell'aggiunta, rimozione e sostituzione dei nodi nel \textit{cluster}. Può essere un \textit{hypervisor}, nel caso i nodi siano virtuali, o un operatore, nel caso i nodi siano server reali. È probabile che ci siano più gestori, soprattutto se i nodi sono geograficamente distanti.
  \item \textbf{Gestore di rete}: Questo componente gestisce le connessioni interne ed esterne al \textit{cluster}. In particolare, si occupa dell'assegnazione degli indirizzi di rete ai nodi (è possibile che ogni nodo abbia più interfacce di rete e quindi più indirizzi IP).
  \item \textbf{Sistema operativo}: Sebbene appartenga al nodo, il sistema operativo deve essere progettato su misura per il \textit{cluster}, poiché anche questo componente è responsabile per la corretta operatività dell'intera infrastruttura. In particolare, dovrà:
    \begin{itemize}
        \item Ottenere gli indirizzi di rete (è possibile siano più di uno per ogni nodo).
        \item Eseguire una diagnostica interna sullo stato della macchina.
        \item Eseguire eventuali programmi o daemon responsabili delle funzionalità del \textit{cluster} e degli applicativi.
    \end{itemize}
\end{itemize}


\section{Obiettivi}
Come già accennato, il nostro obiettivo è realizzare un \textit{cluster} che sia allo stesso tempo affidabile e semplice da configurare/installare.

In particolare, il sistema dovrà:
\begin{itemize}
    \item Permettere la scalabilità del servizio aumentando o diminuendo il numero di nodi senza interrompere i servizi forniti.
    \item Garantire la disponibilità del servizio anche in caso di guasti sia a livello fisico (\textit{server}, rete), sia a livello \textit{software} (\textit{crash} del sistema operativo o dell'applicativo stesso).
    \item Integrare tool per l'interazione dell'amministratore con il cluster.
    \item Essere privo di \textit{single point of failure}.
    \item Avere tutti i nodi del \textit{cluster} uguali tra di loro a livello strutturale.
\end{itemize}
% TODO: cambierei alcuni nomi dall'inglese all'italiano, quelli che rimangono in inglese vanno messi in conrsivo

\section{Soluzioni}
Per il raggiungimento degli obbiettivi sopra citati sono stati fatti,
al momento della progettazione, degli accorgimenti sui diversi 
elementi che compongono il \textit{cluster}.
Abbiamo deciso di utilizzare \textbf{libvirt} come \textbf{Gestore dei nodi} cosi' da poter astrarre 
dall'\textit{\textbf{hypervisor}} e permettere una maggiore scalabilita' del servizio.
Per la disponibilita' abbiamo deciso di usare il protocollo \textit{\textbf{Raft}} per la
gestione del consenso distribuito. L'omogeneita' dei nodi e' garantita dal protocollo appena 
citato. Per finire sono state anche scritte due \textbf{CLI} (Command Line Interface): 
una di controllo/configurazione e una di utilizzo finale.
La prima server al sistemista per gestire il cluster mentre la seconda serve all'utente finale 
per usufruire delle feature offerte dal cluster.
        
% Sezione vuota per scaletta

\section{Environment configuration}
To use the cluster correctly, it is first necessary to configure the 
environment responsible for its instantiation. Specifically, the following
steps are required:
\begin{itemize}
	\item Install \textit{libvirt}, ensuring the user has the necessary permissions,
    along with other required applications.
	\item Create the system image that will be used to boot the virtual
    machines (VMs).
\end{itemize}
To automate this process, we developed a Bash script that configures 
everything needed on Arch Linux-based distributions.

The reason for this restriction is that the image creation process uses 
an Arch Linux-exclusive tool called \textit{Archiso}\cite{4}. 
This tool allows the creation of system images and is typically used 
to generate customized installation Live environments.

Using \textit{Archiso}, we organized the filesystem, configured services, selected 
installed packages, and set up the bootloader.

Our objective was to minimize the image size by removing unnecessary programs 
and/or replacing them with lighter alternatives. We also chose \textit{Syslinux} 
as the bootloader due to its lightweight nature compared to the standard 
\textit{GRUB}, setting a boot timeout of zero to reduce startup times. By default, the bootloader has an average wait timeout of 5 seconds.
This delay allows users to select an operating system or access the BIOS menu.
However, in our case, this functionality is unnecessary, and the 5-second delay is 
simply wasted time.
To eliminate this delay, we modify the configuration file 
\begin{verbatim}
env_setup/iso_creation/syslinux/syslinux.cfg
\end{verbatim}
by adding the line:
\begin{verbatim}
PROMPT 0
\end{verbatim}
This change ensures that the bootloader skips the UI entirely and immediately loads the default 
operating system (Arch Linux in our case).
While this modification is not strictly required for system functionality, 
its simplicity makes it a worthwhile optimization, reducing boot time from over 5 seconds to 
less than 1 second per node.


Once the image is created, it is copied into the default 	\textit{libvirt} directories
(\texttt{/var/lib/libvirt/images/}) so that \textit{libvirt} can use it during 
the VM creation process.

When the Bash script is executed, it performs the following steps:
\begin{itemize}
	\item Installs the required packages for virtualization.
	\item Grants the current user the necessary permissions to use \textit{libvirt} 
    without requiring root privileges.
  \item Uses \textit{Archiso} to create the system image.
\end{itemize}
      Initially, we considered using a precompiled image, but we opted for this 
      approach because it is significantly more secure and eliminates the need to 
      manually track system versioning. Additionally, it provides a simple update 
      process for the cluster. Specifically, updating the cluster only requires modifying 
      the \textit{Archiso} configuration and re-executing the setup script to apply 
      the desired updates.


\section{PIB}
\subsection{Components}
\subsubsection{Nodes}
As previously mentioned, the cluster nodes are Virtual Machines (VMs) created using \textit{libvirt} with an XML file
(\texttt{vm\_creation/} \texttt{sources/vm.xml}) that defines their characteristics.
In \textit{libvirt}, an XML file determines the structure of a specific VM. To make this element flexible, we added 
the following parameters to the XML configuration:
\begin{itemize}
  \item \texttt{RAFT\_NODE\_NAME}: Specifies the name of the VM to be created.
  \item \texttt{PATH\_DISK}: Specifies the virtual disk to be assigned to the VM.
\end{itemize}
Additionally, the path to the system image that must be used when booting the VM is specified as 
\texttt{/var/lib/}\texttt{libvirt/images/}\\ \texttt{raft\_live\_install.iso}. This image was created and saved during the 
environment configuration phase.

\subsubsection{Node and network manager}
We designed the network architecture of the cluster using two distinct subnets, each with specific 
roles to ensure separation between public client-facing traffic and internal cluster communication:
\begin{itemize}
	\item \textbf{Public Subnet (192.168.2.0/24)}:
    This subnet is configured with NAT, meaning nodes in the public network can access the external 
    internet and communicate with external systems but cannot directly communicate with each other within the subnet.
    Within this range, IP addresses 1 and 255 are reserved for the hypervisor and broadcast 
    messages, respectively, while the remaining IPs are dynamically assigned to virtual machines that require external access.

  \item \textbf{Private Subnet (10.0.0.0/24)}:
    This subnet is dedicated exclusively to intra-cluster communication. It is implemented using 
    a virtual bridge, which allows all nodes in the subnet to directly communicate with each other. 
    However, this network has no external internet access, ensuring its sole purpose is to facilitate 
    communication among nodes for cluster-related tasks.
    Similar to the public subnet, IP addresses 1 and 255 are reserved, while the remaining range 
    is available for cluster nodes.
\end{itemize}
By separating the public and private subnets, we ensure that client requests and intra-cluster messages
are routed correctly without interference. The public subnet handles client interactions, while the 
private subnet is used for internal operations, such as coordination and state replication among nodes,
as required by the Raft protocol.
This separation enhances security, performance, maintainability, and scalability of the cluster.

\subsubsection{Operating system}
In our cluster, each virtual machine uses Arch Linux as the operating system. Upon boot, two key 
services are automatically started on each node:
\begin{itemize}
  \item \textbf{\texttt{discovery.service}}:
    This service facilitates the discovery of the IP addresses of other nodes within the 10.0.0.x/24 
    subnet. It performs periodic network scans using nmap. Every 30 seconds, the discovery process 
    retrieves and stores the public and private network interface IPs of other nodes. This continuous 
    scanning ensures that each node remains "aware" of other nodes in the subnet, maintaining seamless communication.

  \item \textbf{\texttt{raft\_daemon.service}}:
    This service is responsible for running the code that implements the cluster's functionality and 
    synchronization. Its main tasks include:
    \begin{itemize}
      \item Mounting and formatting the disk provided to the VM as \texttt{ext4}. This ensures a consistent
        state at every boot, regardless of previous events.
      \item Cloning the precompiled code version from the repository via git clone. This approach, 
        combined with the branch structure previously described, allows for hot updates to nodes 
        without requiring VM reconstruction.
      \item Checking if \texttt{discovery.service} has detected at least one other node in the network. 
        If no other node is found, the service waits until the condition is met. This ensures 
        that the program only executes after at least one network scan has been completed.
    \end{itemize}
    The last task seems a limitation, but it's crucial to prevent a node from starting execution 
    without considering the presence of other nodes in the network. However, this restriction is 
    not impactful, as the Raft protocol requires a cluster to have at least five nodes to ensure
    stability following failures (the minimum number for a PIB cluster). 
    If only one node exists in the network, delaying its availability does not limit the protocol
    or the cluster. Once at least two nodes are present, \texttt{raft\_daemon.service} executes the program 
    fetched from the repository, making the node fully available to the cluster.
\end{itemize}


\subsection{RPC Module}
The implementation of an RPC (Remote Procedure Call) system was a fundamental aspect of our project. 
Given the importance of this module, we developed a custom solution tailored to our specific needs. 
It is worth noting that the use of the acronym RPC here is somewhat improper: unlike traditional RPCs, 
which are synchronous and allow procedures to be invoked on other machines, our messages are 
asynchronous (the sender does not wait for a response). These messages trigger a response behavior 
on the remote node rather than directly invoking a procedure. Our decision to use the term RPC 
was simply to maintain consistency with the terminology in the reference article \cite{1}.

\subsubsection{Goals:\\}
The primary objectives of our RPC module were as follows:
\begin{itemize}
	\item Provide a common interface for all RPCs to enable uniform handling without requiring individual management for each message type.
	\item Implement a standardized system for data serialization and deserialization.
	\item Introduce a system that allows adding, modifying, or removing one or more RPCs without significantly impacting the code logic.
\end{itemize}

\subsubsection{Solutions:\\}
To build this module, we relied on a widely used mechanism for serializing Abstract Data Types 
(ADTs): \textit{Protobuf}. An existing implementation of this mechanism was utilized \cite{5}.

The following steps outline our approach to achieving the defined goals:
\begin{itemize}
	\item \textbf{Common interface for RPC messages}:
    We created a unified interface that applies to all RPC messages, defining a set of 
    core functions for each message:
    	\begin{itemize}
        	\item ToString: Returns the state of the message and its values in string format.
        	\item Execute: Executes the behavior defined by the RPC. To achieve this, the 
            module leverages cluster metadata for node information, cluster configurations 
            (e.g., the number of nodes), and sender details.
        	\item Encode: Serializes the message into bytes.
        	\item Decode: Deserializes a byte array into a message, populating the message with the extracted data.
	    \end{itemize}
    	Each RPC message serves as a wrapper around its corresponding Protobuf message.
      %FIX: not very clear

    	\item \textbf{Scalability and modularity}:
    The modular design of the system proved highly effective in meeting our objectives. 
    During development, we encountered the need to add new message types beyond those 
    defined by the Raft protocol (e.g., AppendEntryRpc, AppendResponse, RequestVoteRpc, 
    RequestVoteResponse). Examples include RPCs used to notify nodes with voting rights 
    or to handle messages received from the client.
    These requirements emerged during development rather than during initial planning, 
    and the structured design of the custom RPC module ensured that such additions 
    were minimally invasive and low-impact.

    	\item \textbf{Automated generation of new RPCs}:
    Scalability was further enhanced through the use of a script and an RPC template, 
    which allowed for the generation of new RPCs. This approach also helped prevent 
    potential errors when creating new message types.

    	\item \textbf{External conversion module}:
    When a message is received, it is first deserialized into a generic message type 
    containing a TYPE and a PAYLOAD.
    The TYPE indicates which RPC was received and determines the decoding method to apply.
    Once the type is verified, the payload is extracted and converted into the appropriate internal format.

    This design was necessary because it is not possible to automatically convert a raw 
    byte message directly into its corresponding Protobuf message. Each Protobuf message 
    has its own decoding method.

    While we acknowledge this as a technical debt, since adding a new RPC requires updating 
    the conversion module, we deemed it essential to establish a general format understood 
    by all nodes. Once the generic message is received, each node proceeds to extract and 
    internally process the payload.
\end{itemize}

\subsection{Raft for distributed consensus}
Raft is a distributed consensus algorithm designed to enable a group of computers (nodes) 
to agree on a shared state, even in the presence of faults or network latency. In essence, 
Raft ensures that all nodes maintain an identical and up-to-date copy of the data, thereby 
guaranteeing the consistency and reliability of a distributed system.

\subsubsection{Motivations}
We chose this protocol to manage consensus among the nodes for the following reasons:
\begin{itemize}
	\item \textbf{Homogeneity of nodes}:
    As outlined in our objectives, the cluster must consist of identical nodes. Raft 
    facilitates this requirement seamlessly, as the protocol inherently assumes and 
    enforces uniformity among nodes.
	\item \textbf{High fault tolerance}:
    The cluster is designed to avoid any single point of failure. Raft, by its 
    very definition, is free of such vulnerabilities, making it a highly reliable choice 
    for our cluster. This feature aligns with our goal of achieving a fault-tolerant and 
    dependable distributed system.
	\item  \textbf{Dynamic node configuration}:
    Raft supports dynamic cluster reconfiguration without requiring restarts or manual 
    intervention. While this capability is not strictly essential for protocol application, 
    it significantly enhances scalability, allowing us to flexibly increase or 
    decrease the number of nodes in the cluster as needed.
\end{itemize}

\subsubsection{Limitations}
The structure described above allows us to implement all our needs while maintaining a 
reasonable level of decoupling between components. However, it presents some significant limitations:
\begin{itemize}
    \item \textbf{Complexity}: The described system is highly complex in both implementation 
      and comprehension. While it allows for rapid updates, understanding its workings 
      for making modifications is not straightforward.
    \item \textbf{Performance}: One of the primary issues with our Raft application is that, 
      as will be explained in later sections, it is not resource-efficient.
    \item \textbf{Log Compression}: Although not strictly necessary, log compression is 
      highly useful. However, our use of the protocol does not account for this functionality. 
      This means that if compression needs to be implemented in the future, there is no 
      guarantee that the current structure will support it.
\end{itemize}

\subsubsection{Implementation:\\}
The following discussion pertains exclusively to the Raft component. Other aspects will be 
described in subsequent sections.

It should be noted that no form of \textbf{SpinLock} or \textbf{BusyWait} was implemented 
in the code. Whenever the need arose, a loop was created that was suspended through the 
blocking read of a message on a channel.

The Raft protocol implementation comprises the following modules:
\begin{itemize}
  \item \textbf{\texttt{raft\_log}}:
    Handles the storage of \texttt{LogEntry} objects and the \texttt{commitIndex}.
    This module defines the Raft logging system, where the "log" represents the 
    collection of entries recording the operations executed in the cluster. The key 
    component is the \texttt{LogInstance}, which represents a single log entry. It consists 
    of the actual entry and a channel used to return the result of the associated 
    operation. Adding this channel was necessary to avoid blocking the system while 
    processing a request's result. Once the result is ready, it is sent through the 
    channel to the corresponding node.

    The logging system also includes the \textbf{\texttt{LogEntrySlave}} and \textbf{\texttt{LogEntryMaster}}. Both share
    the same log, but the former has read-only permissions, while the latter has 
    both read and write permissions.

    When adding a new log entry, the \textbf{\texttt{AppendEntry}} function checks whether the entry
    already exists in the log. If not, it appends the entry to the log. Adding a 
    log entry does not affect the \textbf{\texttt{commitIndex}}, which remains unchanged. During 
    the entire insertion process, a mutex is used to prevent data races caused 
    by concurrent entry insertions.
    
    The \texttt{\textbf{IncreaseCommitIndex}} function is used to increment the commit index, and 
    both operations are restricted to the \textbf{\texttt{LogEntryMaster}}. The purpose of \textbf{\texttt{LogEntrySlave}}
    will become clearer when discussing configuration changes.

  \item \textbf{\texttt{nodeIndexPool}}: This module stores the addresses of active nodes in the network 
    and tracks the last \texttt{LogEntry} they received. It is used only by the leader to determine 
    the content of \texttt{AppendEntry} messages for each node.

  \item \textbf{\texttt{ClusterMetadata}}: This module describes both the cluster and the node itself, 
    such as the node's IP address, the leader's IP (both public and private), the term used
    to track leadership validity, the node's role, timer information (election and heartbeat),
    and details about the last election, including quorum, the node for which it voted, 
    and its voting rights.

  \item \textbf{\texttt{ConfPool}}: This module manages system configurations. It tracks the main 
    configuration, which represents the current system state, as well as one or more temporary 
    configurations that might be under evaluation or preparation to become the new main configuration.
  
    This module is the only component with write access to \textbf{\texttt{raft\_log}}, while configuration 
    instances are granted read-only permissions.
  
    The reason for this is straightforward: during a transition phase (\textbf{joint configuration}) 
    from one configuration to another, neither configuration should be allowed to make 
    unilateral decisions about the cluster's state independently. Therefore, if a new \texttt{ConfPool}
    entry needs to be added, the module will append the new 
    entry to the log and notify the active configurations that there is a new entry ready to be committed.

    Afterward, it will wait for both configurations to be ready to commit the entry. Only
    then will the \texttt{commitIndex} of the log be incremented. Since entries are always committed
    in the order they are received, it is impossible for an entry to be committed before an
    earlier one, ensuring both arrival order and log consistency across the cluster nodes.

    This process is the same for both the leader and followers. The distinction lies in how
    each instance of the configuration declares readiness to commit an entry, depending on
    whether the node is a Follower or a Leader.
    
    Specifically:
    \begin{itemize}
      \item Follower: Automatically marks the entry as committed as soon as it is received
        and notifies the ConfPool that the next entry can be committed.
      \item Leader: Sends an AppendEntry to all nodes in the configuration. Once a majority
        of the followers in that configuration notify the leader that they have committed 
        the entry, the leader will notify the \texttt{ConfPool} that the commit can proceed.
    \end{itemize}

    If two configurations are active simultaneously, the procedure described above is
    applied to both configurations. The commit only occurs when both configurations notify
    the \texttt{ConfPool} that it can proceed.
    Notifications are sent via channels to a \texttt{ConfPool} goroutine, which waits on configuration
    channels before incrementing the commitIndex.

    It should be noted that when \texttt{LogEntry} entries are added to the \texttt{ConfPool} log, the process
    blocks only for the part where the entries are added to the log. Subsequently, it 
    simply notifies, via a channel, an internal goroutine that handles the commit procedure 
    for each added entry.
    This way, adding one or more \texttt{LogEntry} entries does not require waiting for the commit 
    procedure to complete before continuing with other operations. As a result, the commit
    procedure is entirely asynchronous with respect to the entry insertion.
    Once an entry is ready to be committed, a message is sent on another channel to wake up 
    a function responsible for applying the entry in question.

    Currently, the described system cannot commit multiple log entries simultaneously; 
    it can only add multiple entries simultaneously. Additionally, using a mutex within 
    the \texttt{AppendEntry} procedure to coordinate insertions is not sufficient.

    Since the handling of each received RPC is managed asynchronously by a goroutine, two 
    or more goroutines might end up waiting on the mutex, creating a race condition over 
    which one inserts its entries first.
    To solve this issue, a synchronization avoidance mechanism is needed to maintain the 
    arrival order of the goroutines. Unfortunately, this has not been implemented yet, 
    as it is considered an optimization to be addressed later, given that the cluster 
    already has a stable and functional version.

    Within the \texttt{ConfPool}, there is another goroutine that is typically suspended, waiting
    on a channel: \textbf{checkNodeToUpdate}. This function is responsible for handling one of 
    the edge cases mentioned in the protocol: the addition of a new node to the cluster 
    after it has already started, ensuring the new node reaches a consistent state.
    To achieve this, whenever an \texttt{AppendEntry} needs to be sent to a node identified by 
    its IP, the system verifies if the node is present on the network. If the node 
    is not present, its IP is stored in a vector. When and if the node joins the network, 
    its IP is sent to a channel that wakes up the function mentioned above. This 
    function verifies whether the newly added node needs to be updated and, if so, 
    calls an RPC to revoke its voting rights, as per the protocol. Once the node 
    reaches a consistent state, its voting rights are restored using the same RPC. 

  \item \textbf{CommonMatchIndex}: This defines an interface responsible for monitoring
    the last index on which all nodes agree. To achieve this, it periodically performs 
    a series of checks to ensure that the nodes are aligned and that the system can 
    proceed consistently. Additionally, it carries out different procedures depending 
    on each node's voting rights:
    \begin{itemize}
      \item If a node has voting rights, a new goroutine is launched to periodically 
        check for updates to its matchIndex.
      \item If a node does not have voting rights, a goroutine periodically checks 
        whether its \texttt{matchIndex} has surpassed the \texttt{CommonMatchIndex}. Once this condition
        is met, the goroutine shifts to monitoring updates to the \texttt{matchIndex} of the corresponding node.
    \end{itemize}
    To update the \texttt{CommonMatchIndex}, the system verifies that the new \texttt{matchIndex} is 
    greater than both the current\\ \texttt{CommonMatchIndex} and the node's current \texttt{matchIndex}. If 
    this condition is met, the interface increments the count of "stable" nodes, i.e., 
    nodes with an updated index. 
    If the count of stable nodes exceeds half the nodes in the configuration, the entry
    associated with the index can be considered committed, and the \texttt{CommonMatchIndex} is incremented.

    This system is essential for individual configurations to notify the configuration 
    manager that they have committed a new entry. This allows the configuration manager 
    to proceed with incrementing the global commit index. 

    Finally, this entire procedure is performed exclusively by the leader to update the commit index.
\end{itemize}

In its implementation, several RPCs are defined using the previously described module. 
Below are the definitions:
\begin{itemize}
  \item \textbf{AppendEntryRpc}: This is sent by the leader on two occasions: when it needs
    to send a heartbeat signal to the follower nodes or when there are entries in the log 
    that still need to be replicated.
    When a follower receives this message, it first checks the sender’s term. If the term 
    is lower, the replication operation fails, and it sends an AppendResponse with a negative 
    result. If the term passes the check, the node changes its role to follower and sets 
    leaderIp to the sender’s address. This is necessary for two main reasons:
    \begin{itemize}
      \item Only the leader can send AppendEntryRpc, so the sender must be the leader.
      \item There is a possibility that the receiver is also a leader but of a lower 
        term (an outdated leader), which must be demoted to follower. This situation 
        could occur if the old leader went offline and a new election took place, 
        resulting in an increased term for the remaining nodes.
    \end{itemize}
    Afterward, the node checks the length of the entries vector received to determine 
    if the message is a heartbeat message or an attempt by the leader to replicate its log. 
    In the latter case, it performs a consistency check that may fail under the following 
    circumstances:
    \begin{itemize}
      \item The follower has fewer entries than the leader, excluding the new entries 
        (in this case, the leader must decrement the \texttt{nextIndex} and \texttt{matchIndex} for this 
        node and retry the send).
      \item At the same position (index), the follower and leader have different terms. 
    \end{itemize}
    If either of these conditions occurs, the node responds with false, including 
    the index of the inconsistent entry. Otherwise, it responds with true, indicating success.
        
    Finally, the node resets its election timeout.

  \item \textbf{AppendEntryResponse}: This is sent by each follower node to the leader 
    in response to an AppendEntryRpc. The leader checks the result of the AppendEntryRpc:
    \begin{itemize}
      \item If the result is positive, it increments the node’s \texttt{nextIndex} and \texttt{matchIndex} parameters.
     \item If the result is negative, there are two possible reasons: either the leader’s 
       term is lower than the sender’s term, in which case the leader adjusts its term and 
        reverts to the follower state; or there is a log inconsistency with the sender node, 
        in which case the leader adjusts the corresponding \texttt{nextIndex}. This ensures that in the 
        next AppendEntryRpc, the leader will resolve the inconsistency.
    \end{itemize}

  \item \textbf{VoteRequest}: This is sent by any node whose election timer has expired to request votes 
    from other nodes. When a node receives this, it performs three checks:
    \begin{itemize}
      \item The sender’s term is at least equal to the receiver’s term.
      \item The receiver has either already voted for the sender or has not voted at all.
      \item The index and term of the sender’s last log entry are greater than or equal to those of the receiver.
      \item If all these conditions are met, the receiver grants its vote to the sender; otherwise, it does not.
    \end{itemize}

  \item \textbf{VoteResponse}: This is the response nodes send after receiving a vote request. 
      Depending on the outcome of the request:
      \begin{itemize}
        \item If successful, the receiver increments the number of supporters.
        \item If unsuccessful, it decrements the number of supporters.
      \end{itemize}
      The node calculates the majority threshold. If the number of supporters surpasses 
      the threshold, it changes its role to leader. Otherwise, it remains a follower and 
      resets its election parameters (supporter count and the candidate it voted for).

  \item \textbf{UpdateNode}: This is used by the leader to grant voting rights to a follower node.

  \item \textbf{ClientRequest}: This represents a message received by the leader from a client. The 
    leader translates this RPC into a log entry, adds it to its log, and waits for the 
    cluster to process it. Once processed, the result is sent back to the client. Each 
    log entry is associated with a return channel for the result. The goroutine handling 
    the RPCs waits for this result on the channel.

  \item \textbf{ClientReturnValue}: This represents the message containing the result 
    of a ClientRequest that the leader sends back to the client.

  \item \textbf{NewConf}: This message is sent by the operator to declare the initial 
    configuration of the cluster. 
\end{itemize}

\subsection{Dettagli implementativi dell'applicativo}
Ora che e' chiara la struttura del progetto si puo' procedere con la descrizione del
codice e delle relative funzionalita'.
\subsubsection{Obiettivi}
% Sezione vuota per scaletta
%TODO: ingloberei questa sotto sezione alla sezione Obiettivi

\subsubsection{Struttura del codice}
Il progetto è composto da numerosi moduli di seguito esposti:
\begin{itemize}
  \item node: definisce la struttura per un nodo Raft e si occupa della ricezione e dell'invio di messaggi. 

  \item localFs: definisce l'astrazione del file system locale e le operazioni di scrittura e lettura su di esso.

  \item pkg: ospita il codice generato per la serializzazione/deserializzazione dei messaggi tra i nodi.
    
  \item raft\_log: definisce il sistema di log di Raft, in cui la struttura LogInstance rappresenta una singola voce nel log di Raft. Il pacchetto importa anche una codifica basata su protobuf per serializzare i messaggi 
    di log per la trasmissione in rete. Una caratteristica chiave di questo pacchetto è il canale ReturnValue, che supporta operazioni asincrone come la ricezione di conferme di commit per le voci del log. \\
    Contiene l'implementazione principale del log per Raft, utilizzando un sync.RWMutex per gestire l'accesso concorrente ai log. Questo file definisce la struttura logEntryImp, che gestisce un elenco di voci di log, la 
    dimensione del log e l'indice di commit. \\
    \'E principalmente costituito da LogEntrySlave e LogEntryMaster. Entrambi condividono lo stesso log, ma con permessi diversi: il LogEntrySlave ha solo permessi di lettura, mentre il LogEntryMaster ha sia permessi di scrittura 
    che di lettura. Il compito del master è notificare agli slave quando vengono aggiunte nuove entry. Gli slave, invece, notificano al master quando hanno confermato le nuove voci, e, al ricevimento di tutte le conferme, 
    il master procede con il commit ufficiale delle voci.

  \item rpcs: definisce un'interfaccia generale Rpc, utilizzata in tutto il progetto per astrarre la logica delle varie chiamate RPC nel sistema Raft. \\AppendEntryRpc implementa l'RPC per l'aggiunta delle voci di log. Questa 
    struct 
    contiene la logica che un leader utilizza per inviare voci di log ai propri follower.\\AppendResponse gestisce le risposte agli RPC di aggiunta delle entry. Quando un follower riceve e processa una richiesta di append di una
    voce dal leader, questo pacchetto aiuta a convalidare e riconoscere il successo di quell'operazione. Lavora con i componenti del log e della gestione dello stato per garantire una corretta replica. \\RequestVoteRPC gestisce il
    processo di elezione del leader. Durante le elezioni, questo pacchetto è responsabile della gestione delle richieste di voto inviate dai nodi che cercano di diventare il nuovo leader. Interagisce con lo stato del nodo per 
    garantire che le richieste di voto siano legittime, un aspetto cruciale per mantenere un processo elettorale affidabile all'interno del protocollo Raft. \\RequestVoteResponse.go processa le risposte alle richieste di voto. 
    Questa struct gestisce la decisione di concedere o meno il proprio voto a un candidato, basandosi sullo stato e sul log sia del nodo richiedente che del nodo rispondente. Come il pacchetto RequestVoteRPC, garantisce l'integrità
    dell'elezione del leader. 

  \item clusterMetadata: definisce i metadati relativi al cluster, come il term per tracciare la validità della leadership, le informazioni su quale nodo è l'attuale leader, i dettagli sul quorum delle elezioni e il ruolo che 
    ricopre ciascun nodo.

  \item confPool: gestisce le configurazioni del sistem. Tiene traccia della configurazione principale, che rappresenta lo stato attuale del sistema, e di una o più configurazioni temporanee che potrebbero essere in fase di 
    valutazione o preparazione per diventare la nuova configurazione principale. Inoltre, si assicura che tutti i nodi del sistema siano d'accordo sulla configurazione corrente. Questo è fondamentale per evitare 
    incoerenze e garantire un funzionamento corretto del sistema. Infine, tiene traccia di alcuni parametri importanti (nextIndex e matchIndex) dei log di ciascun nodo, al fine di garantire una corretta replicazione del log. 
  
  \item server: è la componente centrale che viene eseguito su ciascun nodo. Definisce la struttura e il loop principale del lavoro del nodo. Sta in ascolta di messaggi e connessioni in arrivo, delegando la loro gestione alle 
    rispettive componenti elencate prima.

\end{itemize}



\subsubsection{Funzionamento}
% Sezione vuota per scaletta
All'avvio, il nodo legge la configurazione, che include l'indirizzo IP dei nodi pesenti e del nodo stesso. Viene quindi creato un server Raft che rappresenta il nodo stesso. Questa configurazione è fondamentale per definire la 
rete di nodi Raft che formeranno il cluster.
Una volta configurato, il server avvia il nodo Raft, che si mette in ascolto per comunicazioni e richieste. A questo punto, ogni nodo può interagire con gli altri nel cluster, comunicando attraverso RPCs (Remote Procedure Calls).

\textbf{Comunicazione tra Nodi}\\
Una parte cruciale del flusso è la comunicazione tra nodi, che avviene principalmente attraverso due fasi: replicazione del log e elezioni del leader. Queste interazioni vengono gestite tramite vari tipi di RPC definiti nel progetto.
Quando un nodo leader deve replicare le voci del log, invia un'RPC AppendEntryRpc ai nodi follower. I follower, una volta ricevuta la richiesta, e, se la richiesta è valida e il log viene replicato correttamente, i follower 
rispondono confermando la replica.
Durante il processo di elezione del leader, i nodi inviano una richiesta di voto con l'RPC RequestVoteRPC. I nodi che ricevono questa richiesta processano la decisione di votare o meno per il candidato leader, assicurandosi 
che i requisiti per il voto (come la consistenza del log) siano soddisfatti.

\textbf{Processo di elezione}\\
Il sistema Raft monitora costantemente la presenza di un leader attivo. Quando un nodo non riceve messaggi di "heartbeat" (AppendEntry) dal leader entro un determinato timeout, questo evento segnala che il leader potrebbe essere 
inaccessibile o non funzionante. Ogni nodo follower mantiene un proprio timer, e se scade senza ricevere comunicazioni dal leader, quel nodo considera il leader inattivo e transita in uno stato di candidate.
In questo stato, il nodo invia richieste di voto agli altri nodi del cluster attraverso l'RPC RequestVoteRPC. Il nodo candidato cercherà di ottenere voti sufficienti per diventare leader. Durante questa fase, 
ogni nodo che riceve una richiesta di voto valuta se concedere il proprio voto al candidato, basandosi sulla propria situazione di stato e log.
Se il candidato riceve voti dalla maggioranza dei nodi, assume il ruolo di leader e comincia immediatamente a inviare messaggi AppendEntryRpc ai follower per confermare la propria leadership e mantenere i log sincronizzati. 
Se invece nessun candidato ottiene la maggioranza, viene avviato un nuovo ciclo elettorale fino a quando non viene selezionato un leader. 

\textbf{Replicazione dei Log, Commit e applicazione allo stato della macchina}\\
Uno dei flussi operativi principali riguarda la gestione del log, che è il meccanismo attraverso cui Raft garantisce la consistenza dei dati nei nodi. Il log è gestito da due implementazioni separate: una per il leader e una 
per i follower.
Quando un leader riceve un nuovo comando, lo aggiunge al suo log interno e invia un'RPC AppendEntry ai follower per replicare questo comando. I follower ricevono questo messaggio, lo aggiungono al loro log, e rispondono al leader. 
Il leader tiene traccia di quanti follower hanno replicato con successo l'entry e una volta che una voce è stata replicata con successo su una maggioranza di nodi, il leader può committare la voce. Questo processo è gestito 
attraverso canali asincroni (applyC e NotifyAppendEntryC) definiti nelle implementazioni del log di leader e follower. Il leader invia un commit alla maggior parte dei nodi, e i follower seguono l'ordine di commit. 
Quando le voci vengono committate, esse vengono applicate allo stato condiviso del sistema.
Il canale ReturnValue nel log permette di gestire i risultati delle operazioni in maniera asincrona, in modo che il sistema possa continuare ad operare senza attendere che ogni operazione venga completata in modo sincrono.

\textbf{Gestione delle configurazioni}\\
L'aspetto più importante dell'intero cluster è la gestione delle configurazioni. Il sistema prevede due configurazioni distinte: quella corrente e quella nuova. La presenza simultanea di entrambe indica che il cluster si 
trova in uno stato di transizione da una configurazione all'altra.\\
Quando il leader riceve una richiesta da un client per applicare una nuova configurazione, che include un elenco di nodi, aggiorna l'elenco, imposta la nuova configurazione e informa i nuovi nodi che non sono ancora autorizzati
a votare. Il leader inizia quindi a replicare il proprio log, aggiornando i nuovi nodi follower.\\
Durante questa procedura, potrebbe accadere che uno dei nuovi nodi non venga trovato a causa della sua assenza nella subnet. In tal caso, le informazioni su quel nodo vengono memorizzate e aggiornate in seguito. Infine, 
il sistema esegue controlli periodici per verificare se il log di ciascun nodo è sincronizzato; quando un nodo risulta aggiornato, il leader gli concede il diritto di voto.

\subsubsection{Limiti}
% Sezione vuota per scaletta
Anche se il nostro sistema funziona come descritto nel documento (citare documento), presenta un grave difetto. La Figura 3 illustra la parte del sistema principalmente coinvolta nel processo di \textit{commit} degli indici.
\\
\begin{lstlisting}[language=Go]
func (c *commonMatchImp) 
    checkUpdateNewMatch(sub *substriber) { 
  var halfNodeNum = c.numNodes/2 
  for c.run { 
    var newMatch = <-sub.Snd 
    if newMatch > c.commonMatchIndex && 
        sub.Trd < newMatch { 
      c.numStable++ 
      if c.numStable > uint(halfNodeNum) {
        if c.commitEntryC != nil {
          c.commitEntryC <- c.commonMatchIndex
        } 
        c.commonMatchIndex++
        c.numStable = 1
      } 
    } 
    sub.Trd = newMatch
  } 
}
\end{lstlisting}
Il problema deriva da una sezione critica del codice a cui accedono contemporaneamente più Goroutine senza alcun meccanismo di controllo dell'accesso. In particolare, è possibile che una Goroutine 
superi il controllo iniziale ed esegua le operazioni previste, e che subito dopo una seconda Goroutine superi lo stesso controllo. A questo punto, però, la prima Goroutine ha già modificato i valori condivisi all'interno del 
corpo dell'istruzione if, il che significa che la seconda Goroutine opera su valori non aggiornati, invalidando di fatto la condizione del guard. Anche se abbiamo eseguito numerosi test (che riconosciamo essere insufficienti 
per dimostrare l'assenza di bug) e abbiamo trovato difficile riprodurre le condizioni necessarie per far emergere questa condizione di race, non abbiamo ancora osservato questo problema nella pratica.

%TODO: sposterei questa sezione a dopo o prima della sezione Soluzioni (dipende da cos'è Soluzione)
\section{Tool di configurazione}
\subsection{Obiettivi}
% Sezione vuota per scaletta

\subsection{Utilizzo}
% Sezione vuota per scaletta

\subsection{Limiti}
% Sezione vuota per scaletta


\section{Valutazioni}
\subsection{Raggiungimento obiettivi}
% Sezione vuota per scaletta
Il progetto ha centrato con successo gli obiettivi principali definiti inizialmente, ponendo le basi per un sistema 
cluster affidabile, scalabile e di facile configurazione. Tra i risultati più rilevanti si evidenzia:
\begin{enumerate}
  \item Affidabilità e Tolleranza ai Guasti: L'adozione del protocollo Raft ha permesso di garantire 
  un consenso distribuito tra i nodi, eliminando qualsiasi single point of failure. Questo approccio assicura 
  che, anche in caso di guasti a livello fisico (server, rete) o software (crash del sistema operativo), 
  il cluster continui a funzionare in maniera coerente, mantenendo l'integrità dello stato condiviso.
  
  \item Scalabilità Dinamica: Il sistema consente l'aggiunta o la rimozione di nodi senza interrompere
  i servizi forniti. L'uso di libvirt come gestore dei nodi ha reso possibile l’astrazione dell’hypervisor,
  semplificando la gestione delle macchine virtuali e supportando facilmente la scalabilità del cluster.

  \item Configurazione Automatizzata: La creazione di script bash ha semplificato e automatizzato la fase 
  di setup dell’ambiente e delle immagini di sistema. Questo approccio riduce significativamente 
  i tempi di configurazione e garantisce consistenza nelle installazioni.

  \item Strumenti di Interazione: Lo sviluppo di tool CLI ha introdotto un'interfaccia intuitiva per 
  amministratori e utenti finali. La CLI consente di monitorare lo stato del cluster, gestire 
  configurazioni dinamiche e inviare richieste specifiche, come il recupero o la modifica di file.

  \item Architettura Modulare: La progettazione del sistema è stata effettuata con un focus sulla modularità
  e sull’isolamento delle componenti. Moduli dedicati, come il raft\_log, ConfPool e RPC module, hanno permesso
  di ottenere un sistema organizzato e manutenibile, facilitando future estensioni e miglioramenti.

  \item Ottimizzazione delle Risorse: La scelta di Arch Linux come sistema operativo ha permesso di minimizzare l’uso delle 
  risorse, grazie a un’immagine di sistema snella e personalizzata con componenti leggeri come Syslinux al posto di GRUB.
\end{enumerate}

\subsection{Difficoltà riscontrate}
Inizialmente, la sfida principale consisteva nella gestione di \textit{libvirt}, in particolare nell'
impostazione delle macchine virtuali (VM) e nella configurazione di sottoreti distinte per separare la rete 
pubblica dalla rete privata. A causa della complessità di \textit{libvirt}, comprendere i suoi meccanismi 
intricati richiedese una ripida curva di apprendimento. Come già accennato, il nostro obiettivo era creare un 
ambiente *snello e manutenibile*, ma raggiungere questo risultato si è rivelato tutt'altro che semplice.

Uno degli aspetti più impegnativi è stato progettare e sviluppare un sistema distribuito. Questo compito è 
notoriamente difficile, infatti sistemi di questo callibro presentano una complessità 
intrinseca e dei requisiti stringenti come, per esempio, la consistenza, la resistenza ai guasti e la 
coordinazione tra nodi. A questo si aggiunge \textit{Raft}:
tradurre i suoi principi e le sue linee guida presentate nell'articolo (citare articolo) in una soluzione 
concreta ha richiesto una quantità di tempo considerevole ed è stata, simultaneamente, una sfida impegnativa da 
un punto di vista mentale.
Inoltre, puntavamo a sviluppare un sistema scalabile, con diversi livelli di astrazione e composto da 
componenti scarsamente accoppiati tra loro. Questo era essenziale poiché, 
con la crescita del progetto, la gestione del codice diventava sempre più difficile, portando a un crescente 
debito tecnico. Di conseguenza, abbiamo effettuato diverse revisioni del design architettonico per affrontare 
questi problemi.

Nonostante i nostri sforzi per limitare il debito tecnico, esso è rimasto una sfida costante. La base del codice 
diventava vasta e complessa, il che complicava la gestione delle \textit{goroutine} e di tutti gli aspetti 
relativi alla sincronizzazione tra di esse, e persino il semplice compito di seguire il flusso del sistema. 
Anche i test si sono rivelati difficili a causa di questa complessità. Abbiamo trovato particolarmente 
complicato progettare il sistema di configurazione, poiché non fu immediatamente chiaro come inserirlo nel 
sistema già presente. Abbiamo impiegato tempo e sforzi, sia nella rilettura dell'articolo di riferimento, sia nel
la ristrutturazione del codice, per arrivare ad una soluzione corretta e funzionante

Un altro ostacolo significativo è stato eseguire le operazioni di \textit{testing} sul sistema. Gestire più 
macchine virtuali ed effettuare \textit{debug} tramite semplici stampe  
si è rivelato macchinoso e dispendioso in termini di tempo. Un sistema di \textit{logging} distribuito sarebbe 
stato una 
soluzione più efficace, ma implementare un tale sistema avrebbe richiesto un progetto aggiuntivo.

In retrospettiva, bilanciare la complessità tecnica mantenendo al contempo modularità, scalabilità e 
un'architettura pulita è stata la sfida più grande che abbiamo affrontato durante tutto il progetto.

\subsection{Miglioramenti realizzabili}
Nonostante i risultati significativi, il progetto presenta alcune criticità e aree di miglioramento che 
potrebbero essere affrontate per rendere il sistema più robusto, efficiente e accessibile.
\begin{enumerate}
  \item Gestione dei Log Distribuiti: Attualmente, il debugging del sistema e la raccolta dei 
  log risultano macchinosi, in quanto basati su semplici stampe. L’implementazione di un sistema 
  di logging distribuito permetterebbe di centralizzare la raccolta dei log, facilitando il 
  monitoraggio delle macchine virtuali e l’identificazione dei problemi in tempo reale. 
  
  \item Risoluzione della Data Race: È stata identificata una possibile data race che povocherebbe 
  una race condition all'interno della procedura di commit dei log, dove più goroutine possono accedere 
  contemporaneamente a risorse condivise senza un meccanismo di controllo adeguato. Per 
  risolvere questo problema, si potrebbe implementare una coda di priorità a doppia struttura,
  garantendo così l'ordine di esecuzione delle operazioni senza 
  introdurre ulteriori overhead o spinlock.

  \item Ottimizzazione delle Performance: Sebbene il sistema funzioni correttamente, 
  alcune operazioni del protocollo Raft potrebbero essere ottimizzate ulteriormente. 
  Ad esempio, la compressione dei log non è attualmente supportata, il che può portare 
  a un aumento delle risorse utilizzate nel tempo. Introdurre una strategia di snapshot
  o di compressione permetterebbe di ridurre lo spazio occupato dai log e migliorare 
  le performance complessive.

  \item Automazione del Deployment in Produzione: Anche se è stata realizzata 
  un’automazione per la configurazione iniziale, il deployment del cluster in ambienti 
  di produzione richiede ancora un certo intervento manuale. 

  \item Interfaccia Grafica (GUI): L'uso esclusivo della CLI per interagire con il 
  cluster potrebbe risultare poco accessibile per utenti meno esperti o non tecnici. 
  Lo sviluppo di un’interfaccia grafica semplificata permetterebbe di visualizzare lo stato 
  dei nodi, monitorare le metriche del sistema e gestire le configurazioni con maggiore intuitività.

  \item Testing e Validazione del Sistema: La complessità del sistema ha reso difficoltosa 
  l’esecuzione dei test e la validazione delle funzionalità. Potrebbe essere utile introdurre 
  una suite di test automatizzati, anche se non sarebbero sufficienti a garantire un 
  comportamento corretto.
\end{enumerate}


\end{document}
