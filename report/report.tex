\documentclass[acmtog]{acmart}

\AtBeginDocument{%
  \providecommand\BibTeX{{%
    Bib\TeX}}}

%%
%% These commands are for a JOURNAL article.

%\citestyle{acmauthoryear}
\usepackage[italian]{babel}
\usepackage[T1]{fontenc}
\usepackage{amsmath}
\usepackage{mathtools}
\usepackage{amssymb}
\usepackage{stmaryrd}

\title{PIB: Una soluzione alla complessità di installazione dei cluster moderni}
%%%%%%%%%%%%%%%%%%%%%%%%%%%%%%%%%%%%%%%%%%%%%%%%%%%%%%%%%%%%%%%%%%%%%%%%%%%%%%%%%%%%%%%%%%%%%%%%%%%%%%%%%%%%%%%%%%%%%%%%%%%%%%
%%%%%%%%%%%%%%%%%%%%%%%%%%%%%%%%%%%%%%%%%%%%%%%%%%%%%%%%%%%%%%%%%%%%%%%%%%%%%%%%%%%%%%%%%%%%%%%%%%%%%%%%%%%%%%%%%%%%%%%%%%%%%%
\author{Alberto Damo}
\authornote{Entrambi gli autori hanno contribuito equamente a questo progetto.}
\email{alberto.damo@studenti.unipd.it}
\author{Alessandro Pirolo}
\authornotemark[1]
\email{alessandro.pirolo@studenti.unipd.it}
\affiliation{%
  \institution{Università degli Studi di Padova}
  \city{Padova}
  \state{Veneto}
  \country{Italia}
}

%%
%% By default, the full list of authors will be used in the page
%% headers. Often, this list is too long, and will overlap
%% other information printed in the page headers. This command allows
%% the author to define a more concise list
%% of authors' names for this purpose.
\renewcommand{\shortauthors}{Damo, Pirolo}

%%
%% The abstract is a short summary of the work to be presented in the
%% article.
\begin{abstract}
  A %TODO: Sposterei il preambolo qui
\end{abstract}


%%%%%%%%%%%%%%%%%%%%%%%%%%%%%%%%%%%%%%%%%%%%%%%%%%%%%%%%%%%%%%%%%%%%%%%%%%%%%%%%%%%%%%%%%%%%%%%%%%%%%%%%%%%%%%%%%%%%%%%%%%%%%%
%%%%%%%%%%%%%%%%%%%%%%%%%%%%%%%%%%%%%%%%%%%%%%%%%%%%%%%%%%%%%%%%%%%%%%%%%%%%%%%%%%%%%%%%%%%%%%%%%%%%%%%%%%%%%%%%%%%%%%%%%%%%%%
\begin{document}

\maketitle


\section{Abstract}
The world today is highly interconnected and complex. For this reason, every application 
designed for the general public must ensure reliable service at all times. To address 
these challenges, networked distributed systems, also known as \textit{clusters}, are 
often used. These systems enable scalability and reliability of services. However, they 
are inherently complex and challenging to set up. Consider, for instance, the configuration 
process of a \textit{Kubernetes} or \textit{Ceph} cluster, which is intricate and 
time-consuming unless external tools are used to simplify the task—tools that are not always 
trustworthy.  

This complexity is due, among other factors, to the architecture chosen to model the system, 
typically an \textit{Orchestrator}-based model. While this model offers numerous advantages, 
it comes at a cost: the nodes within the cluster are not homogeneous.  

This seemingly minor detail is, in fact, the root cause of the complexity in configuring, 
updating, and maintaining the cluster.  

In this report, we will describe how we created a cluster that meets the necessary reliability 
requirements while also allowing for quick and straightforward installation and configuration.


\section{Composizione di un cluster}
Prima di poter analizzare i nostri obiettivi, come li abbiamo raggiunti e perché siano state fatte determinate scelte, è necessario descrivere gli elementi che compongono un \textit{cluster} e le loro caratteristiche:

\begin{itemize}
  \item \textbf{Nodi}: Sono i computer che si occupano della computazione necessaria per mantenere il servizio attivo. Possono essere fisici (\textit{server} dedicati) o virtuali (macchine virtuali, \textit{container}).
  \item \textbf{Gestore dei nodi}: Questo componente si occupa dell'aggiunta, rimozione e sostituzione dei nodi nel \textit{cluster}. Può essere un \textit{hypervisor}, nel caso i nodi siano virtuali, o un operatore, nel caso i nodi siano server reali. È probabile che ci siano più gestori, soprattutto se i nodi sono geograficamente distanti.
  \item \textbf{Gestore di rete}: Questo componente gestisce le connessioni interne ed esterne al \textit{cluster}. In particolare, si occupa dell'assegnazione degli indirizzi di rete ai nodi (è possibile che ogni nodo abbia più interfacce di rete e quindi più indirizzi IP).
  \item \textbf{Sistema operativo}: Sebbene appartenga al nodo, il sistema operativo deve essere progettato su misura per il \textit{cluster}, poiché anche questo componente è responsabile per la corretta operatività dell'intera infrastruttura. In particolare, dovrà:
    \begin{itemize}
        \item Ottenere gli indirizzi di rete (è possibile siano più di uno per ogni nodo).
        \item Eseguire una diagnostica interna sullo stato della macchina.
        \item Eseguire eventuali programmi o daemon responsabili delle funzionalità del \textit{cluster} e degli applicativi.
    \end{itemize}
\end{itemize}


\section{Obiettivi}
Come già accennato, il nostro obiettivo è realizzare un \textit{cluster} che sia allo stesso tempo affidabile e semplice da configurare/installare.

In particolare, il sistema dovrà:
\begin{itemize}
    \item Permettere la scalabilità del servizio aumentando o diminuendo il numero di nodi senza interrompere i servizi forniti.
    \item Garantire la disponibilità del servizio anche in caso di guasti sia a livello fisico (\textit{server}, rete), sia a livello \textit{software} (\textit{crash} del sistema operativo o dell'applicativo stesso).
    \item Integrare tool per l'interazione dell'amministratore con il cluster.
    \item Essere privo di \textit{single point of failure}.
    \item Avere tutti i nodi del \textit{cluster} uguali tra di loro a livello strutturale.
\end{itemize}
% TODO: cambierei alcuni nomi dall'inglese all'italiano, quelli che rimangono in inglese vanno messi in conrsivo

\section{Soluzioni}
Per il raggiungimento degli obbiettivi sopra citati sono stati fatti,
al momento della progettazione, degli accorgimenti sui diversi 
elementi che compongono il \textit{cluster}.
Abbiamo deciso di utilizzare \textbf{libvirt} come \textbf{Gestore dei nodi} cosi' da poter astrarre 
dall'\textit{\textbf{hypervisor}} e permettere una maggiore scalabilita' del servizio.
Per la disponibilita' abbiamo deciso di usare il protocollo \textit{\textbf{Raft}} per la
gestione del consenso distribuito. L'omogeneita' dei nodi e' garantita dal protocollo appena 
citato. Per finire sono state anche scritte due \textbf{CLI} (Command Line Interface): 
una di controllo/configurazione e una di utilizzo finale.
La prima server al sistemista per gestire il cluster mentre la seconda serve all'utente finale 
per usufruire delle feature offerte dal cluster.
        
% Sezione vuota per scaletta


\section{PIB}
\subsection{Elementi}
\subsubsection{Nodi}
% Sezione vuota per scaletta

\subsubsection{Gestore dei nodi e della rete}
% Sezione vuota per scaletta
Abbiamo progettato l'architettura di rete del cluster con due subnet distinte, ciascuna con ruoli specifici per garantire la separazione tra il traffico pubblico rivolto ai client e la comunicazione interna del cluster:

    Subnet Pubblica (192.168.2.0/24): Questa subnet è basata su NAT, il che significa che i nodi nella rete pubblica possono accedere a internet esterno e comunicare con sistemi esterni, ma non possono comunicare direttamente tra loro all'interno della subnet. Il pool di rete pubblica è riservato ai nodi che devono interagire con servizi o client esterni. In questo intervallo, gli indirizzi 1 e 255 sono riservati rispettivamente all'hypervisor e ai messaggi di broadcast, mentre i restanti indirizzi IP sono assegnati dinamicamente alle VM che necessitano di accesso esterno.

    Subnet Privata (10.0.0.0/24): Questa subnet è utilizzata esclusivamente per la comunicazione intra-cluster. L'abbiamo creata utilizzando un bridge virtuale, che consente a tutti i nodi della subnet di comunicare direttamente tra loro. Tuttavia, questa rete non ha accesso a internet esterno, assicurando che il suo unico scopo sia facilitare la comunicazione tra i nodi per compiti correlati al cluster. Come nella subnet pubblica, gli indirizzi 1 e 255 sono riservati, e il resto dell'intervallo è disponibile per i nodi del cluster.

Separando le subnet pubbliche e private, garantiamo che le richieste dei client e i messaggi intra-cluster siano instradati correttamente senza interferenze. La subnet pubblica gestisce le interazioni con i client, mentre la subnet privata è dedicata alle operazioni interne, come il coordinamento e la replica dello stato tra i nodi, come richiesto dal protocollo Raft. Questa separazione non solo migliora la sicurezza e le prestazioni, ma facilita anche la manutenibilità e la scalabilità del cluster.
\subsubsection{Sistema operativo}
% Sezione vuota per scaletta
Nel nostro cluster, ogni VM esegue Arch Linux come sistema operativo. All'avvio, due demoni chiave vengono avviati automaticamente su ogni nodo.

    raft_daemon.service: Questo demone è responsabile dell'esecuzione del codice relativo al protocollo di consenso Raft, che garantisce che il nostro sistema distribuito mantenga la tolleranza ai guasti e la consistenza tra tutti i nodi.

    discovery.service: Questo demone facilita la scoperta degli indirizzi IP degli altri nodi all'interno della subnet 10.0.0.x/24. Lo fa eseguendo scansioni di rete periodiche utilizzando nmap. Ogni 30 secondi, il processo di discovery recupera e memorizza gli indirizzi IP delle interfacce di rete pubbliche e private degli altri nodi. Questa scoperta continua garantisce che ogni nodo rimanga "consapevole" degli altri nodi all'interno della subnet, mantenendo così una comunicazione continua tra loro.


\subsection{Raft per il consenso distribuito}
\subsubsection{Motivazioni}
% Sezione vuota per scaletta

\subsubsection{Applicazione}
% Sezione vuota per scaletta

\subsubsection{Limiti}
% Sezione vuota per scaletta

\subsection{Dettagli implementativi dell'applicativo}
\subsubsection{Obiettivi}
% Sezione vuota per scaletta
%TODO: ingloberei questa sotto sezione alla sezione Obiettivi

\subsubsection{Struttura del codice}
% Sezione vuota per scaletta

\subsubsection{Funzionamento}
% Sezione vuota per scaletta

\subsubsection{Limiti}
% Sezione vuota per scaletta

%TODO: sposterei questa sezione a dopo o prima della sezione Soluzioni (dipende da cos'è Soluzione)
\section{Tool di configurazione}
\subsection{Obiettivi}
% Sezione vuota per scaletta

\subsection{Utilizzo}
% Sezione vuota per scaletta

\subsection{Limiti}
% Sezione vuota per scaletta


\section{Valutazioni}
\subsection{Raggiungimento obiettivi}
% Sezione vuota per scaletta

\subsection{Difficoltà riscontrate}
\subsubsection{Alto livello di concorrenza e distribuzione}
% Sezione vuota per scaletta

\subsubsection{Modularizzazione}
% Sezione vuota per scaletta

\subsubsection{Implementazione del protocollo RAFT}
% Sezione vuota per scaletta

\subsubsection{Organizzazione dei test}
% Sezione vuota per scaletta

\subsubsection{Raccolta dei log}
% Sezione vuota per scaletta

\subsubsection{Automatizzare il processo di deployment del cluster in produzione}
% Sezione vuota per scaletta

\subsubsection{Comprensione del cambio di configurazione del protocollo RAFT}
% Sezione vuota per scaletta

\subsubsection{Riconoscimento dei casi di race condition}
% Sezione vuota per scaletta

\subsection{Vincoli e miglioramenti realizzabili}
% Sezione vuota per scaletta


\end{document}
